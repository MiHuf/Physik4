% DOCUMENT CLASS
\documentclass[11pt]{article}
%PACKAGES
\usepackage[utf8]{inputenc}
\usepackage[ngerman]{babel}
\usepackage[reqno,fleqn]{amsmath}
\setlength\mathindent{10mm}
\usepackage{amstext}
\usepackage{amssymb}
\usepackage{fancyhdr}
\usepackage{units}
\usepackage{comment}  % mehrzeilige Kommentare
\usepackage{cancel} %Brueche kuerzen
\usepackage[table]{xcolor}
\usepackage{tabularx}
\newcolumntype{L}[1]{>{\raggedright\arraybackslash}p{#1}} % linksbündig mit Breitenangabe
\newcolumntype{C}[1]{>{\centering\arraybackslash}p{#1}} % zentriert mit Breiten
% Grafik
\usepackage{graphicx}
\usepackage{subfigure}
\usepackage{wrapfig}
% Zahlenwerte mit Einheiten mittels \unit{Zahlenwert}{Einheit}
\usepackage[thinspace,thinqspace,squaren,textstyle]{SIunits}
% FORMATIERUNG
\usepackage[paper=a4paper,left=25mm,right=25mm,top=25mm,bottom=25mm]{geometry}
\setlength{\parindent}{0cm}
\setlength{\parskip}{1.5mm plus1mm minus1.5mm}
% PAGESTYLE
\pagestyle{fancy}
\setlength\headheight{30pt}
\lhead{Michael Hufschmidt, Mat. Nr. 6436122\\Florian Jochheim, Mat. Nr. 6508131}
\rhead{Übungen zur Physik IV, SoSe 2015\\Blatt 11 zum 06.07.2015}
%MATH SHORTCUTS
\newcommand*{\NN}{\mathbb N}
\newcommand*{\ZZ}{\mathbb Z}
\begin{document}
\begin{comment}
\begin{center}
\begin{tabular}{|L{3cm}|C{1cm}|C{1cm}|C{1cm}|}
\hline
Aufgabe:&\cellcolor{blue!25}24 &\cellcolor{blue!25} 25 &\cellcolor{blue!26} $\sum$\\
\hline
mögliche Punkte: & 5 & 5 & 10 \\
\hline
erreichte Punkte: &  &  & \\
\hline
\end{tabular}
\end{center}
\end{comment}

\subsection*{Aufgabe 24}

\subsubsection*{a)}
Hier auch gleich unsere Formelsammlung für den Spickzettel.
Magnetisierung $\vec M(\vec r, t)$, Magnetfeld $\vec H(\vec r, t)$, und Flußdichte
$\vec B(\vec r, t)$ hängen mit der magnetischen Feldkonstante $\mu_0$ wie folgt zusammen
($n$ ist die Teilchenzahldichte, $F$ = freie Energie):
\begin{align*}
  [\vec m] = [\vec \mu] &= \frac{\text{J}}{\text{T}} = \text{A m}^2 \qquad \text{magnetisches Moment}\\
  \vec M & = \frac{\vec m}{V} = - n \cdot \left(\frac{\partial F}{\partial B} \right)
  \qquad \text{Magnetisierung = magn. Moment pro Volumen}\\
  [M] = [H] &= \frac{\text{A}}{\text{m}} \\
  \vec M &= \chi \cdot \vec H =  (\mu_r - 1) \vec H
    \;;\quad \vec B = \mu_0 (\vec H + \vec M) = \mu_r \mu_0 \vec H \\
  \chi & = \mu_r - 1 = \mu_0 \left(\frac{\partial M}{\partial B} \right)_{T, V}
  = - n \mu_0 \left(\frac{\partial^2 F}{\partial B^2} \right)
  \qquad \text{magnetische Suszeptibilität}\\
  [B] &= \text{T} = \frac{\text{V s}}{\text{m}^2} \\
  [\mu_0] &= \frac{\text{V s}}{\text{A m}} \\
  [\epsilon_0] &= \frac{\text{A s}}{\text{V m}} \\
  \mu_B &=  \frac{e \hbar}{2 m_e} = 9{,}274 \cdot 10^{-24}\frac{\text{J}}{\text{T}}
    = 9{,}274 \cdot 10^{-24}\text{A m}^2  \qquad \text{Bohrsches Magneton}
\end{align*}

\subsubsection*{b)}
\begin{align*}
<\mu> &= \mu_B \cdot \frac{\exp\left(+\frac{\mu_B B}{k_B T}\right) -
  \mu_B\exp\left(-\frac{\mu_B B}{k_B T}\right)}
  {\exp\left(+\frac{\mu_B B}{k_B T}\right) + \exp\left(-\frac{\mu_B B}{k_B T}\right) } \\
\intertext{Ein fcc-Gitter enthält  4 Atome pro Elementarzelle (Volumen $= a^3$), also:}
<M> &= \frac{4 \mu_B}{a^3}\cdot \frac{\exp\left(+\frac{\mu_B B}{k_B T}\right) -
  \mu_B\exp\left(-\frac{\mu_B B}{k_B T}\right)}
  {\exp\left(+\frac{\mu_B B}{k_B T}\right) + \exp\left(-\frac{\mu_B B}{k_B T}\right) } \\
  &= \frac{2 \mu_B}{a^3}\cdot \frac{\exp\left(+\frac{\mu_B B}{k_B T}\right) -
  \mu_B\exp\left(-\frac{\mu_B B}{k_B T}\right)}
  {\cosh \left(-\frac{\mu_B B}{k_B T}\right) }
\intertext{Mit $x := \frac{\mu_B B}{k_B T}$:}
\end{align*}

\subsubsection*{c)}

\subsubsection*{d)}

\subsubsection*{e)}

\subsubsection*{f)}

\subsubsection*{g)}

% \newpage
\subsection*{Aufgabe 25}

\subsubsection*{a)}

\subsubsection*{b)}

\subsubsection*{c)}

\subsubsection*{d)}

\subsubsection*{e)}

\subsubsection*{f)}

% \newpage
\subsection*{Aufgabe 26}
\begin{figure}[h!]
  \centering
  \includegraphics[width=18cm]{./aufgabe26.png}
  \caption{Dem Skript entnommen. Links: a) Ferromagnetismus. Rechts: b) Antiferromagnetismus}
\end{figure}


% \newpage
\newpage
\subsection*{Aufgabe 27}
\begin{wrapfigure}{L}{8cm}
  \centering
  \includegraphics[width=8cm]{aufgabe27.png}
\caption{Dem Skript entnommen.}
\end{wrapfigure}
\subsubsection*{a)}
Sättigungsmagnetisierung $M_S$: Die maximal mögliche Magnetisierung.
Remanenz $M_R$: Die nach Abschalten von $H$ verbleibende Magnetisierung.
Koerzitiv-Feldstärke $H_C$ : Die Feld-stärke (in Gegenrichtung) bei der eine
vorhandene Magnetisierung verschwindet.

\subsubsection*{b)}
"`Leichte"' Richtung (z.B Stabmagnet): Durchgezogene Linie. "`Schwere"' Richtung
(dünner Film mit Magnetisierung senkrecht zur Oberfläche): Gestrichelte Linie:
Keine Hysterese.\newline Warum ???

% \newpage

\end{document}

