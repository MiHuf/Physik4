%  DOCUMENT CLASS
\documentclass[11pt]{article}

%PACKAGES
\usepackage[utf8]{inputenc}
\usepackage[ngerman]{babel}
\usepackage[reqno,fleqn]{amsmath}
\setlength\mathindent{10mm}
\usepackage{amstext}
\usepackage{amssymb}
\usepackage{fancyhdr}
% Zahlenwerte mit Einheiten mittels \unit{Zahlenwert}{Einheit}
\usepackage[thinspace,thinqspace,squaren,textstyle]{SIunits}

% FORMATIERUNG
\usepackage[paper=a4paper,left=25mm,right=25mm,top=25mm,bottom=25mm]{geometry}
\setlength{\parindent}{0cm}
\setlength{\parskip}{1.5mm plus1mm minus1.5mm}

% PAGESTYLE
\pagestyle{fancy}
\setlength\headheight{30pt}
\lhead{Michael Hufschmidt, Mat. Nr. 6436122\\Florian Jochheim, Mat. Nr. 6508131}
\rhead{Übungen zur Physik IV, SoSe 2015\\Blatt 02 zum 20.04.2015}

%MATH SHORTCUTS
\newcommand*{\NN}{\mathbb N}

\begin{document}
\subsection*{Aufgabe 3}

\subsubsection*{a)} Die stationäre Schrödinger-Gleichung eines Elektrons im
(als feldfrei angenommenen) Inneren eines 3-dimensionalen Potatialtopfes mit
unendlich hohen Wändern lautet:
\begin{align*}
-\frac{\hbar ^2}{2 m}  \left( \frac{\partial^2}{\partial x^2} + \frac{\partial ^2}{\partial y^2}
 + \frac{\partial^2}{\partial z^2} \right) \varPsi(x, y, z) = E \cdot \varPsi(x, y, z)
\end{align*}
Mit dem Koordiantenursprung in einer Ecke des Kubus lautet die Lösung welche die
Randbedingungen $\varPsi = 0$ für alle $x, y, z \in \{0, a\}$
(d.h. an den Wänden) erfüllt:
\begin{align*}
  \varPsi(x, y, z) = A \cdot \sin\left(n_x \cdot \frac{\pi x}{a}\right)
  \cdot \sin\left(n_y \cdot \frac{\pi y}{a}\right)
  \cdot \sin\left(n_z \cdot \frac{\pi z}{a}\right)
  \quad \text{mit } n_x, n_y, n_z = 1, 2, 3, \cdots
\end{align*}
Eingesetzt in die Schrödinger-Gleichung liefert das dann die Energie-Eigenwerte:
\begin{align*}
  E_{n_x, n_y, n_z} &= \frac{\pi^2 \hbar^2}{2 m a^2}\left(n_x^2 + n_y^2 + n_z^2 \right)
    \quad \text{mit } n_x, n_y, n_z = 1, 2, 3, \cdots \\
\intertext{Damit ergeben sich die ersten Energie-Zustände zu }
  E_{1, 1, 1} &= \frac{\pi^2 \hbar^2}{2 m a^2} \cdot 3\; ; \qquad
  E_{1, 1, 2} = \frac{\pi^2 \hbar^2}{2 m a^2} \cdot 6\; ; \qquad
  E_{1, 2, 2} = \frac{\pi^2 \hbar^2}{2 m a^2} \cdot 9\; ; \qquad
\end{align*}

\subsubsection*{b)}
Die Energie-Differenz zwischen Grundzustand und erstem angeregtem Zustand ist gleich
der Energie-Differenz zwischen dem ersten und zweiten angeregten Zustand, sie beträgt:
\begin{align}
\label{eq1}
  \Delta E &= \frac{3 \pi^2 \hbar^2}{2 m a^2}
\intertext{Damit lässt sich die Gitterkonstante $a$ berechnen:}
\nonumber
a &= \sqrt{ \frac{3 \pi^2 \hbar^2}{2 m \Delta E}}
\end{align}

\subsubsection*{c)}
Mit \eqref{eq1} lässt die die Absortions-Wellenlänge berechnen:
\begin{align*}
 \hbar \omega &= h f = h \frac{c}{\lambda} = \Delta E = \frac{3 \pi^2 \hbar^2}{2 m a^2} =
 \frac{3 \pi^2}{2 m a^2} \cdot \frac{h^2}{(2 \pi)^2} = \frac{3 h^2}{8 m a^2}\\
 \lambda &= \frac{8 m a^2 c }{3 h}
\end{align*}
Das ergibt mit den Zahlenwerten $m = \unit{9,109}{\cdot 10^{-31}}{\kilogram},\;
c = \unit{2,998}{\cdot 10^8}{\meter/\second},\;
h = \unit{6,626}{\cdot 10^{-34}}{\kilogram \, \meter\squared / \second}$
und $a = \unit{4}{\cdot 10^{-10}}{\meter}$\footnote{Die Einheit $\mathrm{\AA}$
ist in DIN 1301-3 ausdrücklich als nicht mehr zugelassene Einheit aufgeführt.
Ihre Verwendung wurde im "`Gesetz über die Einheiten im Messwesen und die
Zeitbestimmung (Einheiten- und Zeitgesetz - EinhZeitG)"' von 1969 im
geschäflichen Verkehr verboten und kann als Ordnungswidrigkeit mit
Bußgeldern geahndet werden. Sie sollte daher auch im Lehrbetrieb nicht verwendet werden.}:
\begin{align*}
 \lambda &= \frac{8}{3}\cdot\frac{\unit{9,109}{\cdot 10^{-31}}{\kilogram} \cdot
 (\unit{4}{\cdot 10^{-10}}{\meter})^2 \cdot \unit{2,998}{\cdot 10^8}{\meter/\second}}
 {\unit{6,626}{\cdot 10^{-34}}{\kilogram \, \meter\squared / \second}}
 = \unit{1,76}{\cdot 10^{-7}}{\meter} = \unit{176}{\nano\meter}
\end{align*}




\subsection*{Aufgabe 4}

\subsubsection*{a)}

\subsubsection*{b)}

\subsubsection*{c)}

\subsection*{Aufgabe 5}

\subsubsection*{a)}

\subsubsection*{b)}

\end{document}
