%  DOCUMENT CLASS
\documentclass[11pt]{article}

%PACKAGES
\usepackage[utf8]{inputenc}
\usepackage[ngerman]{babel}
\usepackage[reqno,fleqn]{amsmath}
\setlength\mathindent{10mm}
\usepackage{amssymb}
\usepackage{fancyhdr}

% FORMATIERUNG
\usepackage[paper=a4paper,left=25mm,right=25mm,top=25mm,bottom=25mm]{geometry}
\setlength{\parindent}{0cm}
\setlength{\parskip}{1.5mm plus1mm minus1.5mm}

% PAGESTYLE
\pagestyle{fancy}
\setlength\headheight{30pt}
\lhead{Michael Hufschmidt, Mat. Nr. 6436122\\Florian Jochheim, Mat. Nr. 6508131}
\rhead{Übungen zur Physik IV, SoSe 2015\\Blatt 01 zum 13.04.2015}

%MATH SHORTCUTS
\newcommand*{\NN}{\mathbb N}

\begin{document}
\subsection*{Aufgabe 1}

\subsubsection*{a)}
Die Ecken eines Einheitswürfels mögen die Koordinaten
\begin{align*}
&A = \{0,0,0\}, B = \{1,0,0\}, C = \{1,1,0\}, D = \{0,1,0\},\\
&E = \{0,0,1\}, F = \{1,0,1\}, G = \{1,1,1\}, H = \{0,1,1\}
\end{align*}
haben. Verbindet man gegenüber liegende Ecken, so ergeben die Linien $\overline{BD}, \overline{BE}, \overline{BG} \text{ und } \overline{EG}$ die Kanten eines Teraeders, mit der Kantenlänge $a = \sqrt{2}$. Der Mittelpunkt des Würfels ist gleichzeitig
Mittelpunkt des Tetraeders, er hat die Koordinaten
$M = \{\frac{1}{2}, \frac{1}{2}, \frac{1}{2}\}$. Die Richtung der Bindungsvektoren
von dem Atom im Mittelpunkt zu den Atomen an den Ecken ist dann gegeben durch
\begin{align*}
\overrightarrow{MB} &= \left\{\frac{1}{2}, -\frac{1}{2}, -\frac{1}{2}\right\}\\
\overrightarrow{MD} &= \left\{-\frac{1}{2}, \frac{1}{2}, -\frac{1}{2}\right\}\\
\overrightarrow{ME} &= \left\{-\frac{1}{2}, -\frac{1}{2}, \frac{1}{2}\right\}\\
\overrightarrow{MG} &= \left\{\frac{1}{2}, \frac{1}{2}, \frac{1}{2}\right\}\\
\end{align*}
Alle vier Vektoren haben den gleichen Betrag
\begin{align*}
\left|\overrightarrow{MB}\right| = \left|\overrightarrow{MD}\right| =
\left|\overrightarrow{ME}\right| = \left|\overrightarrow{MG}\right| =
\sqrt{\frac{1}{4}+\frac{1}{4}+\frac{1}{4}} = \sqrt{\frac{3}{4}} = \frac{1}{2} \sqrt{3}
\end{align*}
Der Winkel zwischen zwei dieser Vektoren ergibt sich aus dem Skalarprodukt, z.B. der ersten beiden:
\begin{align*}
  \vartheta &= \arccos \left(\frac {\overrightarrow{MB} \cdot \overrightarrow{MD}}
  {\left|\overrightarrow{MB}\right| \cdot \left|\overrightarrow{MD}\right|} \right) =
  \left(\frac{-\frac{1}{4}-\frac{1}{4}+\frac{1}{4}}{\frac{1}{2} \sqrt{3} \cdot \frac{1}{2} \sqrt{3}} \right)
    = \arccos \left(\frac{-\frac{1}{4}}{\frac{1}{4} \cdot  \sqrt{3} \cdot \sqrt{3}} \right)
    \\&= \arccos \left( -\frac{1}{3} \right) = 109,47^{\circ}
\end{align*}





\subsubsection*{b)}

\subsubsection*{c)}

\subsection*{Aufgabe 2}

\subsubsection*{a)}

\subsubsection*{b)}

\subsubsection*{c)}

\end{document}
