%  DOCUMENT CLASS
\documentclass[11pt]{article}

%PACKAGES
\usepackage[utf8]{inputenc}
\usepackage[ngerman]{babel}
\usepackage[reqno,fleqn]{amsmath}
\setlength\mathindent{10mm}
\usepackage{amssymb}
\usepackage{fancyhdr}

% FORMATIERUNG
\usepackage[paper=a4paper,left=25mm,right=25mm,top=25mm,bottom=25mm]{geometry}
\setlength{\parindent}{0cm}
\setlength{\parskip}{1.5mm plus1mm minus1.5mm}

% PAGESTYLE
\pagestyle{fancy}
\setlength\headheight{30pt}
\lhead{Michael Hufschmidt, Mat. Nr. 6436122\\Florian Jochheim, Mat. Nr. 6508131}
\rhead{Übungen zur Physik IV, SoSe 2015\\Blatt 01 zum 13.04.2015}

%MATH SHORTCUTS
\newcommand*{\NN}{\mathbb N}

\begin{document}
\subsection*{Aufgabe 1}

\subsubsection*{a)}Die Ecken eines Einheitswürfels mögen die Koordinaten
\begin{align*}
&A = \{0,0,0\}, B = \{1,0,0\}, C = \{1,1,0\}, D = \{0,1,0\},\\
&E = \{0,0,1\}, F = \{1,0,1\}, G = \{1,1,1\}, H = \{0,1,1\}
\end{align*}
haben. Verbindet man gegenüber liegende Ecken, so ergeben die Linien $\overline{BD}, \overline{BE}, \overline{BG} \text{ und } \overline{EG}$ die Kanten eines Teraeders mit der Kantenlänge $a = \sqrt{2}$. Der Mittelpunkt des Würfels ist gleichzeitig
Mittelpunkt des Tetraeders, er hat die Koordinaten
$M = \left\{\frac{1}{2}, \frac{1}{2}, \frac{1}{2}\right\}$. Die Richtung der Bindungsvektoren
von dem Atom im Mittelpunkt zu den Atomen an den Ecken ist dann gegeben durch
\begin{align*}
\overrightarrow{MB} &= \left[\frac{1}{2}, -\frac{1}{2}, -\frac{1}{2}\right]\\
\overrightarrow{MD} &= \left[-\frac{1}{2}, \frac{1}{2}, -\frac{1}{2}\right]\\
\overrightarrow{ME} &= \left[-\frac{1}{2}, -\frac{1}{2}, \frac{1}{2}\right]\\
\overrightarrow{MG} &= \left[\frac{1}{2}, \frac{1}{2}, \frac{1}{2}\right]
\end{align*}
Alle vier Vektoren haben den gleichen Betrag
\begin{align}
\label{eq-betrag}
\left|\overrightarrow{MB}\right| = \left|\overrightarrow{MD}\right| =
\left|\overrightarrow{ME}\right| = \left|\overrightarrow{MG}\right| =
\sqrt{\frac{1}{4}+\frac{1}{4}+\frac{1}{4}} = \sqrt{\frac{3}{4}} = \frac{1}{2} \sqrt{3}
\end{align}
Der Winkel zwischen zwei dieser Vektoren ergibt sich aus dem Skalarprodukt,
z.B. der ersten beiden Vektoren:
\begin{align*}
  \vartheta &= \arccos \left(\frac {\overrightarrow{MB} \cdot \overrightarrow{MD}}
  {\left|\overrightarrow{MB}\right| \cdot \left|\overrightarrow{MD}\right|} \right) =
  \arccos  \left(\frac{-\frac{1}{4}-\frac{1}{4}+\frac{1}{4}}
    {\frac{1}{2} \sqrt{3} \cdot \frac{1}{2} \sqrt{3}} \right)
    = \arccos \left(\frac{-\frac{1}{4}}{\frac{1}{4} \cdot  \sqrt{3} \cdot \sqrt{3}} \right)
    \\&= \arccos \left( -\frac{1}{3} \right) = 109,47^{\circ}
\end{align*}

\subsubsection*{b)}
\begin{itemize}
  \item [fcc-Gitter:] Die nächsten Nachbarn eines Atoms auf einer Würfelecke sind
  die Atome in der Mitte der Würfelflächen, sie haben einen Abstand von
  $\frac{1}{2} \sqrt{2} \cdot a$  vom Eck-Atom. Durch die Würfelecke gehen
  die drei Ebenen xy, yz und zx, in jeder
  gibt es vier Quadranten. Somit hat das Atom in der Würfelecke $3 \cdot 4 = 12$
  nächste Nachbarn. Auf Grund der Translations-Symmetrie gilt das auch für jedes
  andere Atom. Die Koordinationszahl ist also 12.
    \item [bcc-Gitter:] Die nächsten Nachbarn eines Atoms auf einer Würfelecke sind
  die Atome im Zentrum des Würfels, sie haben einen Abstand von $\frac{1}{2} \sqrt{3} \cdot a$
  vom Eck-Atom. Zu jeder Würfelecke gibt es 8 Oktanten mit je einem nächsten Nachbarn.
  Die Koordinationszahl ist also 8.
  \item[Diamant:] Die Diamantstruktur lässt sich mit Tetraedern analog zu
  Aufgabenteil a) aufbauen, das Atom im Zentrum hat 4 nächste Nachbarn im Abstand
  gemäß \eqref{eq-betrag}. Die Koordinationszahl ist somit 4.
\end{itemize}

\subsubsection*{c)} Wenn sich benachbarte harte Kugeln gerade berühren, muss der
Radius $r$ der Kugeln gleich dem halben minimalen Abstand zweier Gitterpunkte sein.
Rie Raumfüllung ergibt sich dann als Verhältnis der Summe aller $n$ Kugelvolumina
$\frac{4}{3} \pi r^3$ zum Volumen der Elementarzelle.

Die würfelförmige Elementarzelle eines \textbf{Diamantgitters} mit der Kantenlänge $a$
lässt sich aus vier versetzt angeordneten Würfeln mit der Kantenlänge $a/2$ zusammensetzen, welche
Gitterpunkte wie im Aufgabenteil a) enthalten. Der Abstand von einem Eckpunkt zum
nächsten Nachbarn ergibt sich aus \eqref{eq-betrag} zu
$\frac{1}{2} \cdot \sqrt{3} \cdot \frac{a}{2} = \frac{a}{4} \cdot \sqrt{3}$. Zum
gleichen Ergebnis kommt man, wenn man den Abstand der zwei Basistome bei $a \cdot \{0,0,0\}$
und $a \cdot \left\{\frac{1}{4}, \frac{1}{4}, \frac{1}{4}\right\}$ berechnet.
Damit ergibt sich der Kugelradius zu $\frac{a}{8} \cdot \sqrt{3}$.
Diese Kugeln haben ein Volumen von $\frac{4}{3} \pi \left(\frac{a}{8} \sqrt{3}\right)^3 =
\frac{\pi}{384} \cdot 3^{3/2} \cdot a^3$. Die Elementarzelle enthält 8 Kugeln in den
8 Ecken, die jeweils zu 8 weiteren Würfeln gehören, ferner 6 Kugeln auf den Flächen,
die jeweils noch zu 6 weiteren Würfeln gehören und schließlich 4 Kugeln im Inneren.
Das ergibt $n = 8 \cdot \frac{1}{8} + 6 \cdot \frac{1}{2} + 4 = 8$ Kugeln pro Würfel.
Diese füllen ein Volumen von $8 \cdot \frac{\pi}{384} \cdot 3^{3/2} \cdot a^3 =
\frac{\pi}{48} \cdot 3^{3/2} \cdot a^3 = 0,34009 \cdot a^3$. Die würfelförmige
Elementarzelle hat ein Volumen vom $a^3$, somit ergibt sich eine Raumfüllung von
etwa 34\% für das Diamantgitter.

Im \textbf{fcc-Gitter} haben die Gitterpunkte einen Abstand von
$\frac{1}{2} \sqrt{2} \cdot a$ zum nächsten Nachbarn, der Kugelradius ist somit
$r = \frac{1}{4} \sqrt{2} \cdot a$ und das Volumen $\frac{4}{3} \pi \left(\frac{a}{4} \sqrt{2}\right)^3 = \frac{\sqrt{2}}{24} \pi\cdot a^3$. Die Zahl der Kugeln pro
Elementarzelle ist sich ähnlich wie im Diamantgitter, allerdings ohne die 4
Kugeln im Inneren, sie beträgt also 4. Damit ergibt sich eine Raumfüllung von
$\frac{\sqrt{2}}{6} \cdot \pi = 0,74048$ also etwa 74\% für das fcc-Gitter.

\subsection*{Aufgabe 2}
Eine Ebene in einem kubischen Kristallgitter sei beschrieben durch ihre Schnittpunkte mit den Koordinaten-Achsen:
\begin{align*}
\vec{S_1} = 2\cdot \vec{a_1} \qquad \vec{S_2}=2\cdot\vec{a_2}\qquad
\vec{S_3}=3\cdot\vec{a_3}
\end{align*}

Im folgenden ist:
\begin{align*}
(x,y,z)=x\cdot\vec{a_1}+y\cdot\vec{a_2}+z\cdot\vec{a_3}
\end{align*}

\subsubsection*{a)}

Berechnung der Miller Indices
\begin{align*}
h = \frac{p}{2}\qquad k = \frac{p}{2}\qquad l = \frac{p}{3}
\end{align*}
Da $p=kgV(2,2,3)=6$ ergibt sich:
\begin{align*}
h = 3\qquad k = 3\qquad l = 2
\end{align*}

\subsubsection*{b)}

Beweis, dass $(h,k,l)$ senkrecht auf der Ebene steht:
Da es sich um ein kubisches Kristallgitter handelt, gilt:
\begin{align}
\label{eq:orthogonal}
\vec{a_i}\cdot\vec{a_j}\propto \delta_{ij}\qquad
\vec{a_1}\cdot\vec{a_1}=\vec{a_2}\cdot\vec{a_2}=\vec{a_3}\cdot\vec{a_3}
\end{align}
Die Ebene wird aufgespannt durch:
\begin{align*}
(-2,2,0)
(-2,0,3)
\end{align*}
Wir berechnen jeweils die Skalarprodukte mit $(h,k,l)$, wobei wegen \eqref{eq:orthogonal} $(x_1,x_2,x_3)\cdot(y_1,y_2,y_3) = x_1y_1+x_2y_2+x_3y_3$
\begin{align*}
(-2,2,0)\cdot(3,3,2)=-6+6+0=0\\
(-2,0,3)\cdot(3,3,2)=-6+0+6=0
\end{align*}
Da $(h,k,l)$ senkrecht auf den beiden Vektoren steht, die die Ebene aufspannen, steht $(h,k,l)$ auch senkrecht auf der Ebene.

\subsubsection*{c)}

Beweis für die Allgemeinheit der Aussage in b) (Annahme eines kubischen Kristallgitters unverändert):

Allgemeine Ebene beschrieben durch ihre Schnittpunkte mit den Koordinaten- Achsen:
\begin{align*}
\vec{S_1} = m_1\cdot \vec{a_1} \\
\vec{S_2} = m_2\cdot\vec{a_2}\\
\vec{S_3} = m_3\cdot\vec{a_3}
\end{align*}

Dabei sind die Miller-Indices:
\begin{align*}
h = \frac{p}{m_1}\\
k = \frac{p}{m_2}\\
l = \frac{p}{m_3}
\end{align*}
Die Ebene wird aufgespannt durch:
 \begin{align*}
(-m_1,m_2,0)\\
(-m_1,0,m_3)
\end{align*}
Wir berechnen wieder die Skalarprodukte mit $(h,k,l)$ unter Berücksichtigung von \eqref{eq:orthogonal}:
\begin{align*}
(-m_1,m_2,0)\cdot(h,k,l)=(-m_1,m_2,0)\cdot\left(\frac{p}{2},\frac{p}{2},\frac{p}{3}\right)=-p+p=0\\
(-m_1,0,m_3)\cdot(3,3,2)=(-m_1,0,m_3)\cdot\left(\frac{p}{2},\frac{p}{2},\frac{p}{3}\right)=-p+0+p=0
\end{align*}
$(h,k,l)$ steht wieder senkrecht auf beiden Vektoren, die die Ebene aufspannen und somit auch senkrecht auf der Ebene
\end{document}
