\subsection*{Aufgabe 13}
\subsubsection*{a)}
In einem schwachen Potential $V(z) = - V_0 + 2 V_g \cos(G \cdot z)$ mit $G = \frac{2 \pi}{a}$
werde die zeitunabhängige Schrödinger-Gleichung
$\mathcal{H} \Psi = (\frac{p^2}{2 m} + V(z))\Psi = E \Psi$ mit dem Ansatz ebener Wellen gelöst:
\begin{align}
\label{eq-wf}
  \Psi_k(z) &= A \cdot \mathrm e^{i k z} + B \cdot \mathrm e^{i (k-G) z}
  \quad \text{mit} \quad k = \frac{\pi}{a} + \kappa
\end{align}
Also:
\begin{align}
\nonumber
0 &= (\mathcal{H} - E(k)) \Psi_k(z) \\
\nonumber
 &= \left(-\frac{\hbar^2}{2 m} \frac{\partial^2}{\partial z^2}  - V_0 + 2 V_g \cos(G \cdot z) -E(k)\right)
\left (A \cdot \mathrm e^{i k z} + B \cdot \mathrm e^{i (k-G) z} \right) \\
\begin{split}
&= \left\lbrace  A \left[ \left(\frac{\hbar^2 k^2}{2 m} -V_0 - E(k)\right) + 2 V_g \cos(G \cdot z) \right]  \right.\\
& \qquad \left. + B \left[ \left(\frac{\hbar^2 (k-G)^2}{2 m} -V_0 - E(k) \right) + 2 V_g \cos(G \cdot z) \right] \mathrm e^{- i G z} \right \rbrace \mathrm e^{ i k z}\\
\end{split}
\end{align}
Die letzte Gleichung lässt sich nur erfüllen, wenn die geschweifte Klammer verschwindet.
Ferner muss sie für alle $z$ erfüllt sein, also insbesodere für $z = 0$ und für
$z = \frac{a}{2} \; \Leftrightarrow \; G \cdot z = \pi$. Für diese $z$-Werte wird dann:
\begin{align}
\begin{split}
  0 &= A \left(\frac{\hbar^2 k^2}{2 m} -V_0 - E(k) + 2 V_g \right) + B  \left(\frac{\hbar^2 (k-G)^2}{2 m} -V_0 - E(k) + 2 V_g \right) \\
  0 &= A \left(\frac{\hbar^2 k^2}{2 m} -V_0 - E(k) - 2 V_g \right) - B  \left(\frac{\hbar^2 (k-G)^2}{2 m} -V_0 - E(k) - 2 V_g \right)
\end{split}
\end{align}

Weitere Umformungen führen dann zu dem homogenen linearen Gleichungssystem (LGS)
aus dem Aufgabenblatt:
\begin{align}
\label{eq-M1}
\begin{pmatrix}
\frac{\hbar^2k^2}{2m}-V_0-E(k) & V_g\\
V_g & \frac{\hbar^2(k-G)^2}{2m}-V_0-E(k)\\
\end{pmatrix}
\begin{pmatrix}
A\\ B\\
\end{pmatrix}
= \begin{pmatrix}
0\\ 0\\
\end{pmatrix}
\end{align}
Oder mit den Abkürzungen
\begin{align*}
& E_{k} := \frac{\hbar^2k^2}{2m}\; ; \quad E_{k-G} := \frac{\hbar^2(k-G)^2}{2m}\; ;
\quad \text{und}\quad E'(k) := E(k)+V_0
\end{align*}
\begin{align}
&\begin{pmatrix}
\label{eq-M2}
E_k-E'(k) & V_g\\
V_g & E_{k-G}-E'(k)\\
\end{pmatrix}
\begin{pmatrix}
A\\ B\\
\end{pmatrix}
= \begin{pmatrix}
0\\ 0\\
\end{pmatrix}
\end{align}
Das LGS ist nur dann nicht trivial lösbar, wenn die Determinante der Matrix
von \eqref{eq-M1} oder \eqref{eq-M2} verschwindet. Mit dieser Bedingung lässt
sich $E'(k)$ berechnen:

\begin{align*}
&\begin{vmatrix}
E_k-E'(k) & V_g\\
V_g & E_{k-G}-E'(k)\\
\end{vmatrix} = 0\\
\Leftrightarrow\quad& (E_k-E')(E_{k-G}-E')-V_g^2 = 0\\
\Leftrightarrow\quad& E'^2-(E_k+E_{k-G})E+E_kE_{k-G}-V_g^2 = 0\\
\Leftrightarrow\quad& E'_\pm =\frac{E_k+E_{k-G}}{2} \pm \sqrt{\left(\frac{E_k+E_{k-G}}{2}\right)^2+V_g^2-E_kE_{k-G}}\\
\Leftrightarrow\quad& E'_\pm =\frac{E_k+E_{k-G}}{2} \pm \sqrt{\left(\frac{E_k-E_{k-G}}{2}\right)^2+V_g^2} \\
\Leftrightarrow\quad& E'_\pm =\frac{\hbar^2}{4m}(k^2+(k-G)^2) \pm \sqrt{\frac{1}{4}\left(\frac{\hbar^2}{2m}\right)^2(k^2-(k-G)^2)^2+V_g^2}\\
\begin{split}
 \Leftrightarrow\quad& E'_\pm =\frac{\hbar^2}{4m}\left(\left(\frac{\pi}{a}+\kappa\right)^2+\left(\left(\frac{\pi}{a}+\kappa\right)-\frac{2\pi}{a}\right)^2\right) \\
& \qquad\qquad \pm \sqrt{\frac{1}{4}\left(\frac{\hbar^2}{2m}\right)^2\left(\left(\frac{\pi}{a}+\kappa\right)^2-\left(\left(\frac{\pi}{a}+\kappa\right)-\frac{2\pi}{a}\right)^2\right)^2+V_g^2}
\end{split}
\end{align*}
Nebenrechnung:
\begin{align*}
\left(\frac{\pi}{a}+\kappa\right)^2+\left(\left(\frac{\pi}{a}+\kappa\right)-\frac{2\pi}{a}\right)^2
&= \left(\frac{\pi}{a}+\kappa\right)^2 + \left(\frac{\pi}{a}+\kappa\right)^2 - 2 \left(\frac{\pi}{a}+\kappa\right)
\left(\frac{2 \pi}{a}\right) + \left(\frac{2 \pi}{a}\right)^2 \\
& = 2 \left(\frac{\pi^2}{a^2} + \frac{2 \pi}{a}\kappa + \kappa^2 - 2\frac{\pi^2}{a^2} - \frac{2 \pi}{a}\kappa + 2\frac{\pi^2}{a^2}  \right) \\
&= 2 \left( \frac{\pi^2}{a^2} + \kappa^2 \right)
\intertext{und Analog:}
\left(\frac{\pi}{a}+\kappa\right)^2-\left(\left(\frac{\pi}{a}+\kappa\right)-\frac{2\pi}{a}\right)^2
&= \left(\frac{\pi}{a}+\kappa\right)^2 - \left(\frac{\pi}{a}+\kappa\right)^2 + 2 \left(\frac{\pi}{a}+\kappa\right)
\left(\frac{2 \pi}{a}\right) - \left(\frac{2 \pi}{a}\right)^2 \\
& = 4 \frac{\pi}{a}\cdot \kappa
\end{align*}
Damit wird:
\begin{align}
\nonumber
  E'_\pm(k) &=\frac{\hbar^2}{2 m}\left(\left(\frac{\pi}{a}\right)^2 + \kappa^2\right)
  \pm \sqrt{4 \left(\frac{\hbar^2}{2 m}\right) \left(\frac{\pi}{a}\cdot \kappa \right)^2 + V_g^2}\\
\label{eq-Ek}
  & = \frac{\hbar^2}{2 m}\left(\left(\frac{\pi}{a}\right)^2 + \kappa^2\right)
  \pm |V_g|\sqrt{1 + \frac{4}{V_g^2}\left(\frac{\hbar^2}{2 m}\right) \left(\frac{\pi}{a}\cdot \kappa \right)^2 }
\end{align}
Wegen $\kappa \ll \frac{\pi}{a}$ lässt sich die Wurzel entwickeln
$\sqrt{1 + x} \approx 1 + \frac{1}{2}x$ und es gilt:
\begin{align}
  E'_\pm(k) &\approx \frac{\hbar^2}{2 m}\left(\left(\frac{\pi}{a}\right)^2 + \kappa^2\right)
  \pm  \left(|V_g| + \frac{2}{|V_g|}\left(\frac{\hbar^2}{2 m}\right) \left(\frac{\pi}{a}\cdot \kappa \right)^2 \right)
\end{align}

Für den Koeffizienten $B$ der Wellenfunktion \eqref{eq-wf} ergibt sich mit \eqref{eq-M1} bzw. \eqref{eq-M2}:
\begin{align}
\nonumber
0 &= \left(\frac{\hbar^2k^2}{2m}-V_0-E(k)\right)\cdot  A + V_g \cdot B =
V_g \cdot A + \left(\frac{\hbar^2(k-G)^2}{2m}-V_0-E(k)\right) \cdot B \\
\Rightarrow \quad & B = A \cdot \frac{\frac{\hbar^2k^2}{2m}-V_0-E(k)-V_g}
   {\frac{\hbar^2(k-G)^2}{2m}-V_0-E(k)-V_g}
= A \cdot \frac{E_k - E'(k)-V_g}{E_{k-G} - E'(k)-V_g}
\end{align}

\subsubsection*{b)}
Am Zonenrand ($\kappa = 0$) ergibt sich für $E'_\pm(k)$ nach \eqref{eq-Ek}
mit $\frac{\pi}{a} = k$ und $E_{k-G} = \frac{\hbar^2(\frac{\pi}{a}-\frac{2\pi}{a})^2}{2m} = \frac{\hbar^2(\frac{\pi}{a})^2}{2m} = E_k$:
\begin{align*}
  E'_+ &= \frac{\hbar^2}{2 m} \cdot k^2 + V_g  = E_k + V_g \\
\Rightarrow & \quad  B_+ = A \cdot \frac{E_k - E'_+(k) - V_g}{E_{k-G}  - E'_+(k) - V_g} =
  A \cdot \frac{E_k - E_k - V_g - V_g}{E_{k}  - E_k - V_g - V_g} =
  A \\
\Rightarrow & \quad  \Psi_+ = A \cdot \left(\mathrm e^{i \frac{\pi}{a} z}
  + \mathrm e^{-i \frac{\pi}{a} z}  \right) =2A\cos{\frac{\pi z}{a}}\\
  E'_- &= \frac{\hbar^2}{2 m} \cdot k^2 - V_g = E_k - V_g \\
\Rightarrow & \quad  B_- = A \cdot \frac{E_k - E'_-(k) - V_g}{E_{k-G}  - E'_+(k) - V_g} =
  A \cdot \frac{E_k - E_k + V_g - V_g}{E_{k}  - E_k + V_g - V_g} =
  = A \\
\Rightarrow & \quad  \Psi_- = 2A\cos{\frac{\pi z}{a}}
\end{align*}

\subsubsection*{c)}
Wir wählen für das Elektron eine Energie $\tilde E$ in der Mitte der Bandlücke.
Dafür gilt dann mit \eqref{eq-Ek}:
\begin{align*}
\tilde E(k) &= \frac{1}{2} (E'_+(k) + E'_-(k)) =
  \frac{\hbar^2}{2 m}\left(\left(\frac{\pi}{a}\right)^2 + \kappa^2\right) =
  \frac{\hbar^2 k^2}{2 m} + \frac{\hbar^2 \kappa^2}{2 m} \\
  \Rightarrow  \quad \kappa &= \frac{\sqrt{2 m}}{\hbar} \sqrt{\tilde E - \frac{\hbar^2 k^2}{2 m}}
\end{align*}
Da $\frac{\hbar^2 k^2}{2 m} > \tilde E$, ist der Radikand negativ, $\kappa$ somit imaginär.
Die Wellenfunktion \eqref{eq-wf} wird dann mit einem rellen $\tilde \kappa$:
\begin{align}
   \Psi^{(1)}_k(z) &= \left( A \cdot \mathrm e^{i\pi\frac{z}{a}} +
     B \cdot \mathrm e^{i\pi\frac{z}{a}}\cdot \mathrm e^{i G z}\right)\cdot \mathrm e^{\tilde \kappa z}
     \quad \text{mit} \quad \tilde \kappa :=\frac{\sqrt{2 m}}{\hbar} \sqrt{\frac{\hbar^2 k^2}{2 m} - \tilde E}
\end{align}
Sie steigt mit wachsendem $z$ exponentiell an.

\subsubsection*{d)}
Für das gegebene Problem lautet die Wellenfunktion:
\begin{align*}
  \Psi(z) &= \begin{cases}
    \Psi^{(1)}(z) \quad\text{für}\quad z < 0\\
    \Psi^{(2)}(z) \quad\text{für}\quad z \ge 0\\
  \end{cases}
\intertext{mit}
\Psi^{(1)}(z) &= \left( A \cdot \mathrm e^{i\pi\frac{z}{a}} +
     B \cdot \mathrm e^{i\pi\frac{z}{a}}\cdot \mathrm e^{i G z}\right)\cdot \mathrm e^{\tilde \kappa z} \\
\Psi^{(2)}(z) &= \mathrm e^{- q z}\quad\text{mit}\quad q = \frac{1}{\hbar}\sqrt{2 m (V_0 - E)}
\end{align*}
Sie steigt exponentiell an bis $z = 0$ und fällt dann exponentiell wieder ab,
hat also ein Maximum bei $z = 0$. Da die Aufenthaltswahrscheinlichkeit propotional
zu $\Psi^*(z) \cdot \Psi(z)$ ist, ist diese ebenfalls bei $z = 0$ maximal. Elektronen
der Energie $\tilde E$ werden also fast ausschließlich an der Oberfläche bzw.
am rechten Rand der linearen Kette sein.





