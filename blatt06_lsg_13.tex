\subsection*{Aufgabe 13}
\subsubsection*{a)}
In einem schwachen Potential $V(z) = - V_0 + 2 V_g \cos(G \cdot z)$ mit $G = \frac{2 \pi}{a}$
werde die zeitunabhängige Schrödinger-Gleichung
$\mathcal{H} \Psi = (\frac{p^2}{2 m} + V(z))\Psi = E \Psi$ mit dem Ansatz ebener Wellen gelöst:
\begin{align*}
  \Psi_k(z) &= A \cdot \mathrm e^{i k z} + B \cdot \mathrm e^{i (k-G) z}
\end{align*}
Also:
\begin{align}
\nonumber
0 &= (\mathcal{H} - E) \Psi_k(z) =
\left(-\frac{\hbar^2}{2 m} \frac{\partial^2}{\partial z^2}  - V_0 + 2 V_g \cos(G \cdot z) -E\right)
\left (A \cdot \mathrm e^{i k z} + B \cdot \mathrm e^{i (k-G) z} \right) \\
\begin{split}
&= \left\lbrace  A \left[ \left(\frac{\hbar^2 k^2}{2 m} -V_0 - E \right) + 2 V_g \cos(G \cdot z) \right]  \right.\\
& \qquad \left. + B \left[ \left(\frac{\hbar^2 (k-G)^2}{2 m} -V_0 - E \right) + 2 V_g \cos(G \cdot z) \right] \mathrm e^{- i G z} \right \rbrace \mathrm e^{ i k z}\\
\end{split}
\end{align}
Die letzte Gleichung lässt sich nur erfüllen, wenn die geschweifte Klammer verschwindet.
Ferner muss sie für alle $z$ erfüllt sein, also insbesodere für $z = 0$ und für
$z = \frac{a}{2} \; \Leftrightarrow \; G \cdot z = \pi$. Für diese $z$-Werte wird dann:
\begin{align*}
  0 &= A \left(\frac{\hbar^2 k^2}{2 m} -V_0 - E + 2 V_g \right) + B  \left(\frac{\hbar^2 (k-G)^2}{2 m} -V_0 - E + 2 V_g \right) \\
  0 &= A \left(\frac{\hbar^2 k^2}{2 m} -V_0 - E - 2 V_g \right) - B  \left(\frac{\hbar^2 (k-G)^2}{2 m} -V_0 - E - 2 V_g \right)
\end{align*}

??? Aber wie kommt man nun zu dem homogenen Gleichungssystem ???

Das homogene Gleichungssystem
\begin{align*}
\begin{pmatrix}
\frac{\hbar^2k^2}{2m}-V_0-E & V_g\\
V_g & \frac{\hbar^2(k-G)^2}{2m}-V_0-E\\
\end{pmatrix}
\begin{pmatrix}
A\\
B\\
\end{pmatrix}
=0\\
\end{align*}
ist nur dann nicht trivial lösbar, wenn die Determinante der Matrix verschwindet.
Mit den Abkürzungen $E_{k} = \frac{\hbar^2k^2}{2m}$, $E_{k-G} = \frac{\hbar^2(k-G)^2}{2m}$
und $E'=E +V_0$ lässt sich mit dieser Bedingung $E'$ berechnen:
\begin{align*}
&\begin{vmatrix}
E_k-E' & V_g\\
V_g & E_{k-G}-E'\\
\end{vmatrix} = 0\\
\Leftrightarrow\quad& (E_k-E')(E_{k-G}-E')-V_g^2 = 0\\
\Leftrightarrow\quad& E'^2-(E_k+E_{k-G})E+E_kE_{k-G}-V_g^2 = 0\\
\Leftrightarrow\quad& E'_\pm =\frac{E_k+E_{k-G}}{2} \pm \sqrt{\left(\frac{E_k+E_{k-G}}{2}\right)^2+V_g^2-E_kE_{k-G}}\\
\Leftrightarrow\quad& E'_\pm =\frac{E_k+E_{k-G}}{2} \pm \sqrt{\left(\frac{E_k-E_{k-G}}{2}\right)^2+V_g^2} \\
\Leftrightarrow\quad& E'_\pm =\frac{\hbar^2}{4m}(k^2+(k-G)^2) \pm \sqrt{\frac{1}{4}\left(\frac{\hbar^2}{2m}\right)^2(k^2-(k-G)^2)^2+V_G^2}\\
\begin{split}
 \Leftrightarrow\quad& E'_\pm =\frac{\hbar^2}{4m}\left(\left(\frac{\pi}{a}+\kappa\right)^2+\left(\left(\frac{\pi}{a}+\kappa\right)-\frac{2\pi}{a}\right)^2\right) \\
& \quad\quad \pm \sqrt{\frac{1}{4}\left(\frac{\hbar^2}{2m}\right)^2\left(\left(\frac{\pi}{a}+\kappa\right)^2-\left(\left(\frac{\pi}{a}+\kappa\right)-\frac{2\pi}{a}\right)^2\right)^2+V_g^2}
\end{split}
\end{align*}
\subsubsection*{b)}

\subsubsection*{c)}

\subsubsection*{d)}

