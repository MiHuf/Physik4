\subsection*{Aufgabe 13}
\subsubsection*{a)}
Das Gleichungssystem:
\begin{align*}
\begin{pmatrix}
\frac{\hbar^2k^2}{2m}-V_0-E & V_G\\
V_G & \frac{\hbar^2(k-G)^2}{2m}-V_0-E\\
\end{pmatrix}
\begin{pmatrix}
A\\
B\\
\end{pmatrix}
=0\\
\end{align*}
Ist genau dann eindeutig lösbar, wenn die Determinante der Matrix verschwindet.
Mit den Abkürzungen $E_{k} = \frac{\hbar^2k^2}{2m}$, $E_{k-G} = \frac{\hbar^2(k-G)^2}{2m}$
und $E'=E +V_0$ lässt sich mit dieser Bedingung $E'$ berechnen:
\begin{align*}
&\begin{vmatrix}
E_k-E' & V_G\\
V_G & E_{k-G}-E'\\
\end{vmatrix} = 0\\
\Leftrightarrow\quad& (E_k-E')(E_{k-G}-E')-V_G^2 = 0\\
\Leftrightarrow\quad& E'^2-(E_k+E_{k-G})E+E_kE_{k-G}-V_G^2 = 0\\
\Leftrightarrow\quad& E'_\pm =\frac{E_k+E_{k-G}}{2} \pm \sqrt{\left(\frac{E_k+E_{k-G}}{2}\right)^2+V_G^2-E_kE_{k-G}}\\
\Leftrightarrow\quad& E'_\pm =\frac{E_k+E_{k-G}}{2} \pm \sqrt{\left(\frac{E_k-E_{k-G}}{2}\right)^2+V_G^2} \\
\Leftrightarrow\quad& E'_\pm =\frac{\hbar^2}{4m}(k^2+(k-G)^2) \pm \sqrt{\frac{1}{4}\left(\frac{\hbar^2}{2m}\right)^2(k^2-(k-G)^2)^2+V_G^2}\\
\begin{split}
 \Leftrightarrow\quad& E'_\pm =\frac{\hbar^2}{4m}\left(\left(\frac{\pi}{a}+\kappa\right)^2+\left(\left(\frac{\pi}{a}+\kappa\right)-\frac{2\pi}{a}\right)^2\right) \\
& \quad\quad \pm \sqrt{\frac{1}{4}\left(\frac{\hbar^2}{2m}\right)^2\left(\left(\frac{\pi}{a}+\kappa\right)^2-\left(\left(\frac{\pi}{a}+\kappa\right)-\frac{2\pi}{a}\right)^2\right)^2+V_G^2}
\end{split}
\end{align*}
\subsubsection*{b)}

\subsubsection*{c)}

\subsubsection*{d)}

