\subsection*{Aufgabe 21}

\subsubsection*{a)}
In Aufgabe 15 d) haben wir für den Fermi-Wellenvektor $k_F$ eines fcc-Kristalls mit der
Gitterkonstanten $a$ bestehend aus Atomen der Wertigkeit $Z$ im Modell
den folgenden Ausdruck hergeleitet:
\begin{align}
\label{eq:k_F} k_F^2 =
  \left(\frac{3 Z}{2 \pi} \right)^{2/3} \cdot \left(\frac{2 \pi}{a} \right)^2
\end{align}

Wir gehen von freien Elektronen aus und somit von Fermikuglen. Eine Extremalbahn
hat dann die Form eines Kreises mit dem Radius $k_F$ und der Fläche
\begin{align}
 A_k = \pi k_F^2 \label{eq:A_k}
\end{align}
womit sich für die Oszillationsperiode$
 \Delta \left(\frac 1B\right)$
ergibt:
\begin{align}
\label{eq:D_B}
 \Delta \left(\frac 1B\right) = 2\pi\frac{e}{\hbar A_k} \overset{\eqref{eq:A_k}}{=} 2\frac{\cancel{\pi}e}{\cancel{\pi} \hbar k_F^2} \overset{\eqref{eq:k_F}}{=}
 2\frac{e}{\hbar \left(\frac{3 Z}{2 \pi} \right)^{2/3} \cdot \left(\frac{2 \pi}{a} \right)^2}
\intertext {Für Cu mit Z = 1 wird das:}
\nonumber
\end{align}

\subsubsection*{b)}
Zu einer gegebenen Oszillationsperiode können wir die dazugehörige
eingeschlossene Fläche $A_k$ über \eqref{eq:D_B} ausdrücken als:
\begin{align}
 A_k= 2\pi \frac{e}{\hbar \Delta \left(\frac 1B\right)}
\end{align}
womit wir für die gemessenen Kurven folgende Werte erhalten:
\begin{align*}
 \begin{tabular}{|c|c|c}
  \hline
  $\Delta \left(\nicefrac 1B\right)$ in $10^{-5}T$ & $A_k$ \\
   \hline
  1.557 & \\
  \hline
  3.80 & \\
  \hline
  1.608 & \\
  \hline
  38 & \\
  \hline
 \end{tabular}
\end{align*}

Aus dem Modell freier Eletronen haben wir mit \eqref{eq:A_k} $A_k = \pi k_F^2$ erwartet.
Die Fermi Fläche von Kupfer ist aber gegenüber der Fermikugel stark verformt.
Wir erhalten verschiedene Bauchbahnen (Bahnen 1-3) die dem Modell freier Elektronen am nächsten kommen.
In diesen Bereichen ist die Fermi Fläche von Kupfer nur leicht
verformt und es kommt somit nur zu vergleichweise kleinen Korrekturen.
In den Bereichen, wo die Fermi-Kugel dem Brillouin-Zonenrand sehr nahe kommt bilden sich
Hälse aus (siehe Bild auf dem Übungsblatt); die Umlaufbahnen in den Hälsen sind
sehr viel kleiner als die in den Bäuchen, weshalb sich viel größere Oszillationen ergeben
($\approx$ 1 Größenordnung). Dies entspricht gerade dem 4. gemessenen Wert.

\subsubsection*{c)}
Die Probe befinde sich in einem inhomogenen Magnetfeld dessen Stärke zwischen
$B - \delta B$ und $B + \delta B$ variiert, das führt dann zu einer Überlagerung
von entsprechend verschobenen Sägezahn-Spektren. Der Effekt verschwindet vollständig,
wenn gilt:
\begin{align*}
  & \frac{1}{B_n} = \frac{1}{B_{n-1}} +  \Delta \left(\frac 1B\right) \overset{!}=
  \frac{1}{B_{n-1} - \delta B} = \frac{1}{B_{n-1} \left(1 - \frac{\delta B_{n-1}}{B_{n-1}}\right)}
  \approx \frac{1}{B_{n-1}} \left(1  + \frac{\delta B}{B_{n-1}} \right)\\
  & \qquad \Leftrightarrow \quad \Delta \left(\frac 1B\right) =
    \frac{1}{B_{n-1}} \cdot \frac{\delta B}{B_{n-1}}
    \quad \Rightarrow \quad B \cdot \Delta \left(\frac 1B\right) = \frac{\delta B}{B}
\end{align*}
Für einen sichtbaren Effekt muss die Inho1mogenität $\delta B / B$ also deutlich
kleiner als $B \cdot \Delta \left(\frac 1B\right)$ sein, das bedeutet für
$B=5\,$T z.B. in der $<$100$>$-Richtung $\delta B / B \ll 7{,}8 \cdot 10^{-5}$.
