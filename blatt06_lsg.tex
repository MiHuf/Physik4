% DOCUMENT CLASS
\documentclass[11pt]{article}
%PACKAGES
\usepackage[utf8]{inputenc}
\usepackage[ngerman]{babel}
\usepackage[reqno,fleqn]{amsmath}
\setlength\mathindent{10mm}
\usepackage{amstext}
\usepackage{amssymb}
\usepackage{fancyhdr}
\usepackage{units}
% Grafik
\usepackage{graphicx}
\usepackage{subfigure}
\usepackage{wrapfig}
% Zahlenwerte mit Einheiten mittels \unit{Zahlenwert}{Einheit}
\usepackage[thinspace,thinqspace,squaren,textstyle]{SIunits}
% FORMATIERUNG
\usepackage[paper=a4paper,left=25mm,right=25mm,top=25mm,bottom=25mm]{geometry}
\setlength{\parindent}{0cm}
\setlength{\parskip}{1.5mm plus1mm minus1.5mm}
% PAGESTYLE
\pagestyle{fancy}
\setlength\headheight{30pt}
\lhead{Michael Hufschmidt, Mat. Nr. 6436122\\Florian Jochheim, Mat. Nr. 6508131}
\rhead{Übungen zur Physik IV, SoSe 2015\\Blatt 06 zum 01.06.2015}
%MATH SHORTCUTS
\newcommand*{\NN}{\mathbb N}

\begin{document}

\subsection*{Aufgabe 13}
\subsubsection*{a)}
Das Gleichungssystem:
\begin{align*}
\begin{pmatrix}
\frac{\hbar^2k^2}{2m}-V_0-E & V_G\\
V_G & \frac{\hbar^2(k-G)^2}{2m}-V_0-E\\
\end{pmatrix}
\begin{pmatrix}
a\\
b\\
\end{pmatrix}
=0\\
\end{align*}
Ist genau dann eindeutig lösbar, wenn die Determinante der Matrix verschwindet. Mit den Abkürzungen $E_{k} = \frac{\hbar^2k^2}{2m}$, $E_{k-G} = \frac{\hbar^2(k-G)^2}{2m}$ und $E'=E +V_0$ lässt sich mit Hilfe dieser Bedingung nach E' auflösen:
\begin{align*}
\begin{vmatrix}
E_k-E' & V_G\\
V_G & E_{k-G}-E'\\
\end{vmatrix} = 0\\
\Leftrightarrow (E_k-E')(E_{k-G}-E')-V_G^2 = 0\\
\Leftrightarrow E'^2-(E_k+E_{k-G})E+E_kE_{k-G}-V_G^2 = 0\\
\Leftrightarrow E'_\pm =\frac{E_k+E_{k-G}}{2} \pm \sqrt{\left(\frac{E_k+E_{k-G}}{2}\right)^2+V_G^2-E_kE_{k-G}}\\
\Leftrightarrow E'_\pm =\frac{E_k+E_{k-G}}{2} \pm \sqrt{\left(\frac{E_k-E_{k-G}}{2}\right)^2+V_G^2} \\
\Leftrightarrow E'_\pm =\frac{\hbar^2}{4m}(k^2+(k-G)^2) \pm \sqrt{\frac{1}{4}\left(\frac{\hbar^2}{2m}\right)^2(k^2-(k-G)^2)^2+V_G^2}\\
\Leftrightarrow E'_\pm =\frac{\hbar^2}{4m}((\frac{\pi}{a}+\kappa)^2+((\frac{\pi}{a}+\kappa)-\frac{2\pi}{a})^2) \pm \sqrt{\frac{1}{4}\left(\frac{\hbar^2}{2m}\right)^2((\frac{\pi}{a}+\kappa)^2-((\frac{\pi}{a}+\kappa)-\frac{2\pi}{a})^2)^2+V_G^2}
\end{align*}
\subsubsection*{b)}

\subsubsection*{c)}

\subsubsection*{d)}



\subsection*{Aufgabe 14}
\subsubsection*{a)}
Die Reziproken Gittervektoren sind definiert durch:
\begin{align*}
\vec{b}_1 = \frac{2\pi}{V} \vec{a}_2\times\vec{a}_3\\
\vec{b}_2 = \frac{2\pi}{V} \vec{a}_3\times\vec{a}_1\\
\vec{b} = \frac{2\pi}{V} \vec{a}_1\times\vec{a}_2\\
\end{align*}
Für ein fcc-Gitter mit den primitiven Basisvektoren
\begin{align*}
\vec{a}_1=\frac{a}{2}\begin{pmatrix}0\\1\\1\end{pmatrix}\\
\vec{a}_2=\frac{a}{2}\begin{pmatrix}1\\0\\1\end{pmatrix}\\
\vec{a}_3=\frac{a}{2}\begin{pmatrix}1\\1\\0\end{pmatrix}
\end{align*}
ergibt sich damit:
\begin{align*}
\vec{b}_1 &= \frac{2\pi}{V}\left(\frac{a}{2}\begin{pmatrix}1\\0\\1\end{pmatrix}\times\frac{a}{2}\begin{pmatrix}1\\1\\0\end{pmatrix}\right) = \frac{a^2\pi}{2V}\begin{pmatrix}-1\\1\\1\end{pmatrix} = \frac{\pi}{2a}\begin{pmatrix}-1\\1\\1\end{pmatrix} \\
\vec{b}_2 &= \frac{2\pi}{V}\left(\frac{a}{2}\begin{pmatrix}1\\1\\0\end{pmatrix}\times\frac{a}{2}\begin{pmatrix}0\\1\\1\end{pmatrix}\right) = \frac{2\pi}{a}\begin{pmatrix}1\\-1\\1\end{pmatrix}\\
\vec{b}_3 &= \left(\frac{a}{2}\begin{pmatrix}0\\1\\1\end{pmatrix}\times\frac{a}{2}\begin{pmatrix}1\\0\\1\end{pmatrix}\right) = \frac{2\pi}{a}\begin{pmatrix}1\\1\\-1\end{pmatrix}
\end{align*}
Der Abstand $\Gamma X$ ist im reziproken Raum der Abstand von $(0,0,0)$ zu $(1,0,0)$ und somit die Länge von $\vec{b}_1$:
\begin{align*}
\Gamma X = |\vec{b}_1| = \frac{\pi}{2a}\sqrt{3}
\end{align*}

\subsubsection*{b)}
Beim Vergleich mit den Energiebändern von Kupfer (s. Vorlesung) lassen sich die Hochsymmetriepunkte aus dem Verlauf der Bindungsenergie ablesen:
\begin{align*}
\Gamma &\approx 100 \unit{eV}\qquad\text{Plateau maximaler  Intensität auf der linken Seite}\\
X &\approx 176.5\unit{eV} \qquad\text{Tiefpunkt der Bindungsenergie}
\end{align*}
\subsubsection*{c)}
Wir berechnen die Länge des zugehörigen k-Vektors um abzuschätzen in der wievielten
Brillouin-Zone sich die gemessenen Endzustände befinden, hierfür benutzen wir die
Position des X-Punktes aus b) $E_X = 176.5 \mathrm{eV}$. Diese Länge muss dabei ein
Vielfaches des $|\Gamma X|$ Abstandes sein:


\begin{align*}
&n\cdot |\Gamma X| = |\vec{k}| = \frac{\sqrt{2mE}}{\hbar}\\
\Rightarrow & n = \frac{\sqrt{2mE}}{\hbar|\Gamma X|} = 2a\frac{\sqrt{2mE}}{\hbar\pi\sqrt{3}}\\
\Rightarrow & n =  2a\frac{\sqrt{2mE}}{\hbar\pi\sqrt{3}} = 2\cdot3,61\cdot10^{-10}\frac{\sqrt{2\cdot9,11\cdot10^{-31}\cdot176.5\cdot1,602\cdot10^-19}}{1,055\cdot10^{-34}\pi\sqrt{3}} \approx 9
\end{align*}
\subsubsection*{d)}

\subsubsection*{e)}
Bei einer mittleren freien Weglänge von $\lambda = 1\mathrm{nm}$ ist der Anteil des Signals der ersten Atomlage $I_1$ am Gesamtsignal $I_ges$, wobei die Ausdehnung der 1. Atomlage: $a/2 = 1,805\cdot10^{-10} \mathrm{m} = 0,1805\cdot \lambda$ sei:
\begin{align*}
\frac{I_1}{I_ges} = \frac{I_0\int_0^{0,185\lambda}\mathrm{e}^{-\frac{z}{\lambda}}\mathrm{d}z}{I_0\int_0^{\infty}\mathrm{e}^{-\frac{z}{\lambda}}\mathrm{d}z} = \frac{-\lambda(e^{-0,185}-1)}{\lambda} = 1-e^{-0,1805} \approx 0,165
\end{align*}


\end{document}

