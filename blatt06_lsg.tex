% DOCUMENT CLASS
\documentclass[11pt]{article}
%PACKAGES
\usepackage[utf8]{inputenc}
\usepackage[ngerman]{babel}
\usepackage[reqno,fleqn]{amsmath}
\setlength\mathindent{10mm}
\usepackage{amstext}
\usepackage{amssymb}
\usepackage{fancyhdr}
\usepackage{units}
% Grafik
\usepackage{graphicx}
\usepackage{subfigure}
\usepackage{wrapfig}
% Zahlenwerte mit Einheiten mittels \unit{Zahlenwert}{Einheit}
\usepackage[thinspace,thinqspace,squaren,textstyle]{SIunits}
% FORMATIERUNG
\usepackage[paper=a4paper,left=25mm,right=25mm,top=25mm,bottom=25mm]{geometry}
\setlength{\parindent}{0cm}
\setlength{\parskip}{1.5mm plus1mm minus1.5mm}
% PAGESTYLE
\pagestyle{fancy}
\setlength\headheight{30pt}
\lhead{Michael Hufschmidt, Mat. Nr. 6436122\\Florian Jochheim, Mat. Nr. 6508131}
\rhead{Übungen zur Physik IV, SoSe 2015\\Blatt 06 zum 01.06.2015}
%MATH SHORTCUTS
\newcommand*{\NN}{\mathbb N}

\begin{document}

\subsection*{Aufgabe 13}
\subsubsection*{a)}
Das Gleichungssystem:
\begin{align*}
\begin{pmatrix}
\frac{\hbar^2k^2}{2m}-V_0-E & V_G\\
V_G & \frac{\hbar^2(k-G)^2}{2m}-V_0-E\\
\end{pmatrix}
\begin{pmatrix}
a\\
b\\
\end{pmatrix}
=0\\
\end{align*}
Ist genau dann eindeutig lösbar, wenn die Determinante der Matrix verschwindet. Mit den Abkürzungen $E_{k} = \frac{\hbar^2k^2}{2m}$, $E_{k-G} = \frac{\hbar^2(k-G)^2}{2m}$ und $E'=E +V_0$ lässt sich mit Hilfe dieser Bedingung nach E' auflösen:
\begin{align*}
\begin{vmatrix}
E_k-E' & V_G\\
V_G & E_{k-G}-E'\\
\end{vmatrix} = 0\\
\Leftrightarrow (E_k-E')(E_{k-G}-E')-V_G^2 = 0\\
\Leftrightarrow E'^2-(E_k+E_{k-G})E+E_kE_{k-G}-V_G^2 = 0\\
\Leftrightarrow E'_\pm =\frac{E_k+E_{k-G}}{2} \pm \sqrt{\left(\frac{E_k+E_{k-G}}{2}\right)^2+V_G^2-E_kE_{k-G}}\\
\Leftrightarrow E'_\pm =\frac{E_k+E_{k-G}}{2} \pm \sqrt{\left(\frac{E_k-E_{k-G}}{2}\right)^2+V_G^2} \\
\Leftrightarrow E'_\pm =\frac{\hbar^2}{4m}(k^2+(k-G)^2) \pm \sqrt{\frac{1}{4}\left(\frac{\hbar^2}{2m}\right)^2(k^2-(k-G)^2)^2+V_G^2}\\
\Leftrightarrow E'_\pm =\frac{\hbar^2}{4m}((\frac{\pi}{a}+\kappa)^2+((\frac{\pi}{a}+\kappa)-\frac{2\pi}{a})^2) \pm \sqrt{\frac{1}{4}\left(\frac{\hbar^2}{2m}\right)^2((\frac{\pi}{a}+\kappa)^2-((\frac{\pi}{a}+\kappa)-\frac{2\pi}{a})^2)^2+V_G^2}
\end{align*}
\subsubsection*{b)}

\subsubsection*{c)}

\subsubsection*{d)}



\subsection*{Aufgabe 14}
\subsubsection*{a)}

\subsubsection*{b)}

\subsubsection*{c)}

\subsubsection*{d)}

\subsubsection*{e)}


\end{document}

