\subsection*{Aufgabe 25}

\subsubsection*{a)}
In einer halb gefüllten Schale ist in jedem $L$-Orbital jeweils ein Elektron,
also $L = \sum_{m_l = -L}^{+L} m_l  = 0$. Die Spins sind so ausgerichtet, dass $S$
maximal wird (Hund 1), also $S = \frac{1}{2} \cdot (2 L + 1) = L + \frac{1}{2}$.
Damit wird $J = S$.

\subsubsection*{b)}
Ein Elektron weniger liefert $S = \frac{1}{2} \cdot 2 L = L$
und das Orbital mit dem kleinsten $m_l$ ist nicht besetzt:
$L = (\sum_{m_l = - L + 1}^{+ L} m_l) =  L$. Hund 3 liefert dann
$J = | L - S | = 0$.

\subsubsection*{c)}
Es muss $J = 0$, aber $L \ne 0$ und $S \ne 0$ gelten, also im Fall b).
Ansonsten würde der van Vleck-Paramagnetismus vom Spin-Paramagnetismus überlagert.

\subsubsection*{d)}
Nach den Kriterien aus c) sollte $\text{Eu}^{3+}$ van Vleck-Paramagnetismus zeigen.

\subsubsection*{e)}
Bei $\text{Eu}^{3+}$ und $\text{Pm}^{3+}$ tauchen die Orbitale deer f-Schale tief ins
Innere des Atoms ein. Störungsrechnung 2. Ordnung liefert Terme
$\Delta E \propto \frac{< n | V | m >}{E_n - E-0}$, die für $E_n \approx E-0$
große Beiträge liefern.

\subsubsection*{f)}
Die d-Orbiatel ragen weit aus dem eigenen Atom heraus und wechselwirken mit
d-Orbitalen von Nachbaratomen (Kristallfeld-Wechselwirkung).
