\subsection*{Aufgabe 22}

\subsubsection*{a)}
\begin{itemize}
  \item {n-Bereich $x>0$}:
Hier ist $N_A = 0$ und für die Dichte der Elektronen gilt in der Boltzmann-Näherung:
\begin{align}
\nonumber
& n_c(T) = N_c^{eff}(T)\cdot e^{- \frac{E_c - \mu_n}{k_B T}}
\intertext{Bei vollständiger Ionisierung der Störstellen gilt $n_c(T) = N_D$,
  Umformungen liefern dann:}
\nonumber
& k_B T \cdot \ln \left(\frac{N_D}{N_c^{eff}}\right) = - (E_c - \mu_n)\; ; \quad \Rightarrow \quad
  \mu_n = E_c - k_B T \cdot \ln \left(\frac{N_c^{eff}}{N_D}\right)
\intertext{Bei Raumtemperatur $T = 20^\circ$C = 293,15\,K ist $k_B T =$ 25,26\,meV,
mit den Zahlenwerten aus dem Aufgabenblatt ergibt sich dann:}
\label{eq-un}
&\mu_n = E_c - \text{25,26\,meV} \cdot \ln \left(\frac{2{,}8\cdot10^{19}}{1 \cdot10^{17}}\right)
= E_c - \text{142,33\,meV}
\end{align}
  \item {p-Bereich $x<0$}: Analog ist hier $N_D = 0$ und für die Löcherdichte gilt:
\begin{align}
\nonumber
& p_v(T) = P_v^{eff}(T)\cdot e^{- \frac{\mu_p - E_v}{k_B T}} = N_A\\
\nonumber
&\quad \Rightarrow \quad k_B T \cdot \ln \left(\frac{N_A}{P_v^{eff}}\right) = E_v - \mu_p
\; ; \quad \Rightarrow \quad \mu_p = E_v + k_B T \cdot \ln \left(\frac{P_v^{eff}}{N_A}\right)
\intertext{Mit Zahlenwerten:}
\label{eq-up}
&\mu_p = E_v + \text{25,26\,meV} \cdot \ln \left(\frac{1{,}0\cdot10^{19}}{5\cdot10^{16}}\right)
= E_v + \text{133,84\,meV}
\end{align}
\end{itemize}
Bei Störstellenerschöpfung sind die Ladungsträgerdichten und somit die chemischen
Potentiale unabhängig von der Akzeptor-- bzw. Donator--Energie.
Aus \eqref{eq-un} und \eqref{eq-up} ergibt sich für die Diffusionspannung als
Abstand der beiden chemischen Potentiale:
\begin{align*}
V_D = \mu_n - \mu_p = E_c - E_v - \text{276,17\,meV} =
   \text{1,12\,eV} - \text{276,17\,meV} = \text{0,84\,eV}
\end{align*}

\subsubsection*{b)}
Wir nehmen an, dass nur im Intervall $[-w_p, w_n]$ Raumladungen durch nicht
kompensierte Akzeptoren (p-Bereich, $z < 0$) bzw. Donatoren (n-Bereich, $z > 0$)
existieren. Dann können wir für die Ladungsdichte $\rho(z)$ ansetzen:
\begin{align*}
\rho(z) =
  \begin{cases}
    0 & \quad \text{ für } \; z < -w_p \\
    -e N_A& \quad \text{ für } \; z \in [-w_p, 0] \\
    e N_D& \quad \text{ für } \; z \in [0, w_n] \\
    0& \quad \text{ für } \; z > w_n
  \end{cases}
\intertext{Mit der Potential-Gleichung
$\frac{\partial^2}{\partial z^2} \Phi(z) = - \frac{\rho(z)}{\epsilon \epsilon_0}$
und der Elektronen-Energie $V(z) = - e \Phi(z)$ ergibt sich:}
\frac{\partial^2}{\partial z^2} V(z) =
  \begin{cases}
    0 & \quad \text{ für } \; z < -w_p \\
    - \frac{e^2 N_A}{\epsilon \epsilon_0} & \quad \text{ für } \; z \in [-w_p, 0] \\
    + \frac{e^2 N_D}{\epsilon \epsilon_0} & \quad \text{ für } \; z \in [0, w_n] \\
    0& \quad \text{ für } \; z > w_n
  \end{cases}
\intertext{Zweimalige Integration liefert dann (wie schon in der Vorlesung angedeutet) den parabolischen Verlauf:}
V(z) =   \begin{cases}
    V(-\infty) & \quad \text{ für } \; z < -w_p \\
    V(-\infty) - \frac{e^2 N_A}{2 \epsilon \epsilon_0} (w_p + z)^2 & \quad \text{ für } \; z \in [-w_p, 0] \\
    V(+\infty) + \frac{e^2 N_D}{2 \epsilon \epsilon_0} (w_n - z)^2 & \quad \text{ für } \; z \in [0, w_n] \\
    V(+\infty) & \quad \text{ für } \; z > w_n
  \end{cases}
\end{align*}




\subsubsection*{c)}
