\subsection*{Aufgabe 22}

\subsubsection*{a)}
\begin{itemize}
  \item {n-Bereich $x>0$}:
Hier ist $N_A = 0$ und für die Dichte der Elektronen gilt in der Boltzmann-Näherung:
\begin{align}
\nonumber
& n_c(T) = N_c^{eff}(T)\cdot e^{- \frac{E_c - \mu_n}{k_B T}}
\intertext{Bei vollständiger Ionisierung der Störstellen gilt $n_c(T) = N_D$,
  Umformungen liefern dann:}
\nonumber
& k_B T \cdot \ln \left(\frac{N_D}{N_c^{eff}}\right) = - (E_c - \mu_n)\; ; \quad \Rightarrow \quad
  \mu_n = E_c - k_B T \cdot \ln \left(\frac{N_c^{eff}}{N_D}\right)
\intertext{Bei Raumtemperatur $T = 20^\circ$C = 293,15\,K ist $k_B T =$ 25,26\,meV,
mit den Zahlenwerten aus dem Aufgabenblatt ergibt sich dann:}
\label{eq-un}
&\mu_n = E_c - \text{25,26\,meV} \cdot \ln \left(\frac{2{,}8\cdot10^{19}}{1 \cdot10^{17}}\right)
= E_c - \text{142,33\,meV}
\end{align}
  \item {p-Bereich $x<0$}: Analog ist hier $N_D = 0$ und für die Löcherdichte gilt:
\begin{align}
\nonumber
& p_v(T) = P_v^{eff}(T)\cdot e^{- \frac{\mu_p - E_v}{k_B T}} = N_A\\
\nonumber
&\quad \Rightarrow \quad k_B T \cdot \ln \left(\frac{N_A}{P_v^{eff}}\right) = E_v - \mu_p
\; ; \quad \Rightarrow \quad \mu_p = E_v + k_B T \cdot \ln \left(\frac{P_v^{eff}}{N_A}\right)
\intertext{Mit Zahlenwerten:}
\label{eq-up}
&\mu_p = E_v + \text{25,26\,meV} \cdot \ln \left(\frac{1{,}0\cdot10^{19}}{5\cdot10^{16}}\right)
= E_v + \text{133,84\,meV}
\end{align}
\end{itemize}
Bei Störstellenerschöpfung sind die Ladungsträgerdichten und somit die chemischen
Potentiale unabhängig von der Akzeptor-- bzw. Donator--Energie.
Aus \eqref{eq-un} und \eqref{eq-up} ergibt sich für die Diffusionspannung als
Abstand der beiden chemischen Potentiale:
\begin{align*}
V_D = \mu_n - \mu_p = E_c - E_v - \text{276,17\,meV} =
   \text{1,12\,eV} - \text{276,17\,meV} = \text{0,84\,eV}
\end{align*}

\subsubsection*{b)}
Wir nehmen an, dass nur in dem Intervall $[-w_p, w_n]$ der Breite
$w = w_p + w_n$ Raumladungen durch nicht kompensierte Akzeptoren
(p-Bereich, $z < 0$) bzw. Donatoren (n-Bereich, $z > 0$) existieren.
Dann können wir für die Ladungsdichte $\rho(z)$ ansetzen:
\begin{align}
\label{eq-rho}
\rho(z) =
  \begin{cases}
    0 & \quad \text{ für } \; z < -w_p \\
    -e N_A& \quad \text{ für } \; z \in [-w_p, 0] \\
    e N_D& \quad \text{ für } \; z \in [0, w_n] \\
    0& \quad \text{ für } \; z > w_n
  \end{cases}
\intertext{Mit der Poisson-Gleichung
$\frac{\partial^2}{\partial z^2} V(z) = - \frac{\rho(z)}{\epsilon \epsilon_0}$
für das elektrostatische Potential $V(z)$ ergibt sich:}
\label{eq-d2V}
\frac{\partial^2}{\partial z^2} V(z) =
  \begin{cases}
    0 & \quad \text{ für } \; z < -w_p \\
    + \frac{e N_A}{\epsilon \epsilon_0} & \quad \text{ für } \; z \in [-w_p, 0] \\
    - \frac{e N_D}{\epsilon \epsilon_0} & \quad \text{ für } \; z \in [0, w_n] \\
    0& \quad \text{ für } \; z > w_n
  \end{cases}
\intertext{Zweimalige Integration liefert dann (wie schon in der Vorlesung angedeutet) den parabolischen Verlauf:}
\label{eq-V}
V(z) =   \begin{cases}
    V(-\infty) & \quad \text{ für } \; z < -w_p \\
    V(-\infty) + \frac{e N_A}{2 \epsilon \epsilon_0} (w_p + z)^2 & \quad \text{ für } \; z \in [-w_p, 0] \\
    V(+\infty) - \frac{e N_D}{2 \epsilon \epsilon_0} (w_n - z)^2 & \quad \text{ für } \; z \in [0, w_n] \\
    V(+\infty) & \quad \text{ für } \; z > w_n
  \end{cases}
\end{align}
Man verifiziert leicht, dass zweimaliges Differenzieren von \eqref{eq-V}
wieder \eqref{eq-d2V} ergibt. Die beiden Potentiale im Unendlichen sind gerade
die chemischen Potentiale aus Aufgabenteil a). Es müssen nun Stetigkeitsbedingungen
für die elektrische Feldstärke $\vec E(z) = - \vec \nabla V(z)$ und das Potential $V(z)$
im Nullpunkt erfüllt sein:
\begin{align}
\label{eq-s1}
\left. \frac{\partial}{\partial z} V(z) \right|_{z = -0} &\overset{!}{=} \left. \frac{\partial}{\partial z} V(z) \right|_{z = +0} \\
\label{eq-s2}
V(z = -0) &\overset{!}{=} v(z = +0)
\end{align}
Aus \eqref{eq-s1} folgt:
\begin{align*}
  \frac{e N_A}{\epsilon \epsilon_0} (w_p + 0) =  \frac{e N_D}{2 \epsilon \epsilon_0} (w_n - 0)
  \; ; \quad \Rightarrow \quad N_A w_p = N_D w_n\; ; \quad \Rightarrow \quad
  w_n = \frac{N_A}{N_D} w_p
\end{align*}
und aus \eqref{eq-s2} folgt:
\begin{align*}
& \mu_p + \frac{e}{2 \epsilon \epsilon_0} N_A w_p^2 = + \mu_n - \frac{e}{2 \epsilon \epsilon_0} N_D w_n^2
  \; ; \quad \Rightarrow \quad \frac{e}{2 \epsilon \epsilon_0} (N_A w_p^2 + N_D w_n^2) = \mu_n -  \mu_p = V_D\\
& \frac{2 \epsilon \epsilon_0 V_D}{e} = N_A w_p^2  + N_D \left( \frac{N_A}{N_D} w_p \right)^2 =
w_p^2 \left( N_A + \frac{N_A^2}{N_D} \right) = w_p^2 N_A \left(\frac{N_D + N_A}{N_D} \right) \\
& \Rightarrow \quad w_p = \sqrt{\frac{2 \epsilon \epsilon_0 V_D}{e} \cdot  \left(\frac{N_D / N_A}{N_D + N_A} \right)}
\intertext{und analog ergibt sich}
& \qquad \;\; w_n = \sqrt{\frac{2 \epsilon \epsilon_0 V_D}{e} \cdot  \left(\frac{N_A / N_D}{N_D + N_A} \right)}
\end{align*}
Mit den $V_D$ aus Teil a) und den Zahlenwerten $N_A = 5 \cdot 10^{16}\;\text{cm}^{-3}$
und $\;N_D = 10 \cdot 10^{16}\;\text{cm}^{-3}$ aus dem Aufgabenblatt wird dann:
\begin{align*}
& \sqrt{\frac{2 \epsilon \epsilon_0 V_D}{e \cdot(N_D + N_A) }} = \sqrt{\frac{2 \cdot 12 \cdot 8,854\cdot10^{-12}\;\text{As/Vm}
  \cdot 0,84\;\text{V}}{1,602\cdot10^{-19}\;\text{As} \cdot 15\cdot 10^{16}\;\text{cm}^{-3}}}
  = \unit{86,18}{\nano\meter}  \\
& \Rightarrow \quad w_p = \unit{122}{\nano\meter}
\; ; \qquad w_n = \unit{60,9}{\nano\meter}
\; ; \qquad w = w_p +  w_n = \unit{183}{\nano\meter}
\end{align*}

\subsubsection*{c)}
Durch Auffüllen von Löchern im p-Bereich mit Elektronen aus dem n-Bereich entsteht
im p-Bereich eine negative Raumladung, im n-Bereich eine positive. Die elektrische
Feldstärke ist also entgegengesetzt zur z-Achs, d.h. negativ.

Eine externe Spannung mit dem Pluspol am n-Bereich vergrößert dieses Feld, dadurch
vergrößert sich die Verarmungsszone und der Abstand zwischen korrespondierenden
Bändern der p-Zone und der n-Zone: Sperr-Richtung.

Umgekehrt: Eine Spannung mit dem Pluspol am p-Bereich verkleinert das Feld, dadurch
verkleinert sich die Verarmungsszone und der Abstand zwischen korrespondierenden
Bändern der p-Zone und der n-Zone: Durchlass-Richtung.

Siehe auch die schönen Bilder im Skript auf Seite 163.

