% DOCUMENT CLASS
\documentclass[11pt]{article}
%PACKAGES
\usepackage[utf8]{inputenc}
\usepackage[ngerman]{babel}
\usepackage[reqno,fleqn]{amsmath}
\setlength\mathindent{10mm}
\usepackage{amstext}
\usepackage{amssymb}
\usepackage{fancyhdr}
\usepackage{units}

% Grafik
\usepackage{graphicx}
\usepackage{subfigure}
\usepackage{wrapfig}
% Zahlenwerte mit Einheiten mittels \unit{Zahlenwert}{Einheit}
\usepackage[thinspace,thinqspace,squaren,textstyle]{SIunits}
% FORMATIERUNG
\usepackage[paper=a4paper,left=25mm,right=25mm,top=25mm,bottom=25mm]{geometry}
\setlength{\parindent}{0cm}
\setlength{\parskip}{1.5mm plus1mm minus1.5mm}
% PAGESTYLE
\pagestyle{fancy}
\setlength\headheight{30pt}
\lhead{Michael Hufschmidt, Mat. Nr. 6436122\\Florian Jochheim, Mat. Nr. 6508131}
\rhead{Übungen zur Physik IV, SoSe 2015\\Blatt 07 zum 08.06.2015}
%MATH SHORTCUTS
\newcommand*{\NN}{\mathbb N}
\newcommand*{\ZZ}{\mathbb Z}
\begin{document}
\begin{center}
\begin{tabular}{|l|c|c|c|c|}
\hline
Aufgabe:\quad &\quad 15 \quad&\quad 16 \quad&\quad  \quad&\quad $\sum$ \quad\\
\hline
mögliche Punkte: \quad& 7 & 3 &  & 10 \\
\hline
erreichte Punkte: &  &  &  & \\
\hline
\end{tabular}
\end{center}

\subsection*{Aufgabe 15}
Gegeben ist die Dispersionsrelation für Elektronen in einem fcc-Gitter:
\begin{align*}
  E_{h_1, h_2, h_3}(\vec k) &= \frac{\hbar^2}{2 m_e}(\vec k + \vec G_{h_1, h_2, h_3})^2
  \quad \text{mit}\; \vec G_{h_1, h_2, h_3} = \frac{2 \pi}{a}
  \begin{pmatrix} - h_1 + h_2 + h_3 \\  + h_1 - h_2 + h_3  \\ + h_1 + h_2 - h_3 \end{pmatrix}
  \,;\quad h_i \in \ZZ
\end{align*}
Entsprechend der Empfehlung im Skript führen wir reduzierte Größen $\epsilon$ und
$\vec \kappa$ ein:
\begin{align}
\nonumber
&\vec \kappa := \vec k \cdot \left(\frac{2 \pi}{a}\right)^{-1} \,;\quad
  \epsilon(\vec \kappa) := \frac{E(\vec k)}{E_0}  \,;\quad
  E_0 = \frac{\hbar^2}{2 m_e} \left(\frac{2 \pi}{a}\right)^2
\intertext{Damit wird die Dispersionsrelation}
\label{eq-disp1}
&\epsilon_{h_1, h_2, h_3}(\vec k) = (\kappa_x-h_1+h_2+h_3 )^2 +
   (\kappa_y+h_1-h_2+h_3 )^2 + (\kappa_z+h_1+h_2-h_3 )^2
\end{align}

\subsubsection*{a)}
Im reziproken Gitter liegt der $\Gamma$-Punkt im Ursprung, die Brillouin-Zone
in [111]-Richtung (Raumdiagonale) erstreckt sich bis zum L-Punkt mit den
Koordinaten $\frac{1}{2} (\vec b_1 + \vec b_2 + \vec b_3)$. In [111]-Richtung
gilt also stets $k_x = k_y = k_z$ bzw. $\kappa_x = \kappa_y = \kappa_z =: x$
mit möglichen $x$-Werten zwischen 0 und $\frac{1}{2}$.
Damit wird die Dispersionsrelation \eqref{eq-disp1}:
\begin{align}
\nonumber
\epsilon_{h_1, h_2, h_3}(x) &= (x-h_1+h_2+h_3 )^2 + (x+h_1-h_2+h_3 )^2 + (x+h_1+h_2-h_3 )^2 \\
\label{eq-disp2}
& = 3 x^2 + 2 x (h_1+h_2+h_3) -2 (h_1 h_2 + h_2 h_3 + h_3 h_1) + 3 (h_1^2 + h_2^2 + h_3^2)\\
\label{eq-disp3}
& = a x^2 + b x + c;
\end{align}
Für das unterste Band ($n = 1$) ist $h_1 =  h_2 =  h_3 = 0$, damit wird:
\begin{align*}
  \epsilon_{000}(x) = 3 x^2 \,;\quad
  \epsilon_{000}(L) = \epsilon_{000}\left(\frac{1}{2}\right) = \frac{3}{4} \,;\quad
  E_{000}(L) = E_0 \cdot \epsilon_{000}(L) = \frac{3 \hbar^2}{2 m_e} \left(\frac{\pi}{a}\right)^2
\end{align*}
Für die höheren Dispersionszweige haben wir mit einer Tabellenkalkulation alle
$h_i \in \{-1, 0, 1\}$ variiert (27 Kombinationen) jeweils $a, b, c$ gemäß Formel
\eqref{eq-disp2} und \eqref{eq-disp3} ausgewertet, siehe Tabelle \ref{tab1}.
Die Bänder wurden entsprechend dem Wert $\epsilon(0,5)$ und anschließend nach
$\epsilon(0)$ als Bandnummer $n$ sortiert. Aus der Tabelle ist auch der
Entartungsgrad ersichtlich (= 1, 1, 3, 3).
\begin{table}[!ht]
\caption{Teil a) Dispersionszweige und Entartungsgrad}
\begin{center}
\begin{tabular}{|c|c|c|c|c|c|r|r|c|}
\hline
\textbf{h1} & \textbf{h2} & \textbf{h3} & \textbf{a} & \textbf{b} & \textbf{c}
& \multicolumn{1}{c|}{\textbf{e(0)}} & \multicolumn{1}{c|}{\textbf{e(0,5)}} & \textbf{n} \\ \hline
0 & 0 & 0 & 3 & 0 & 0 & 0,00 & 0,75 & 1 \\ \hline
-1 & -1 & -1 & 3 & -6 & 3 & 3,00 & 0,75 & 2 \\ \hline
-1 & 0 & 0 & 3 & -2 & 3 & 3,00 & 2,75 & \multicolumn{ 1}{c|}{3} \\ \cline{ 1- 8}
0 & -1 & 0 & 3 & -2 & 3 & 3,00 & 2,75 & \multicolumn{ 1}{c|}{} \\ \cline{ 1- 8}
0 & 0 & -1 & 3 & -2 & 3 & 3,00 & 2,75 & \multicolumn{ 1}{c|}{} \\ \hline
-1 & -1 & 0 & 3 & -4 & 4 & 4,00 & 2,75 & \multicolumn{ 1}{c|}{4} \\ \cline{ 1- 8}
-1 & 0 & -1 & 3 & -4 & 4 & 4,00 & 2,75 & \multicolumn{ 1}{c|}{} \\ \cline{ 1- 8}
0 & -1 & -1 & 3 & -4 & 4 & 4,00 & 2,75 & \multicolumn{ 1}{c|}{} \\ \hline
0 & 0 & 1 & 3 & 2 & 3 & 3,00 & 4,75 & \multicolumn{ 1}{c|}{5} \\ \cline{ 1- 8}
0 & 1 & 0 & 3 & 2 & 3 & 3,00 & 4,75 & \multicolumn{ 1}{c|}{} \\ \cline{ 1- 8}
1 & 0 & 0 & 3 & 2 & 3 & 3,00 & 4,75 & \multicolumn{ 1}{c|}{} \\ \hline
1 & 1 & 1 & 3 & 6 & 3 & 3,00 & 6,75 & 6 \\ \hline
0 & 1 & 1 & 3 & 4 & 4 & 4,00 & 6,75 & \multicolumn{ 1}{c|}{7} \\ \cline{ 1- 8}
1 & 0 & 1 & 3 & 4 & 4 & 4,00 & 6,75 & \multicolumn{ 1}{c|}{} \\ \cline{ 1- 8}
1 & 1 & 0 & 3 & 4 & 4 & 4,00 & 6,75 & \multicolumn{ 1}{c|}{} \\ \hline
-1 & 0 & 1 & 3 & 0 & 8 & 8,00 & 8,75 & \multicolumn{ 1}{c|}{8} \\ \cline{ 1- 8}
-1 & 1 & 0 & 3 & 0 & 8 & 8,00 & 8,75 & \multicolumn{ 1}{c|}{} \\ \cline{ 1- 8}
0 & -1 & 1 & 3 & 0 & 8 & 8,00 & 8,75 & \multicolumn{ 1}{c|}{} \\ \cline{ 1- 8}
0 & 1 & -1 & 3 & 0 & 8 & 8,00 & 8,75 & \multicolumn{ 1}{c|}{} \\ \cline{ 1- 8}
1 & -1 & 0 & 3 & 0 & 8 & 8,00 & 8,75 & \multicolumn{ 1}{c|}{} \\ \cline{ 1- 8}
1 & 0 & -1 & 3 & 0 & 8 & 8,00 & 8,75 & \multicolumn{ 1}{c|}{} \\ \hline
-1 & -1 & 1 & 3 & -2 & 11 & 11,00 & 10,75 & \multicolumn{ 1}{c|}{9} \\ \cline{ 1- 8}
-1 & 1 & -1 & 3 & -2 & 11 & 11,00 & 10,75 & \multicolumn{ 1}{c|}{} \\ \cline{ 1- 8}
1 & -1 & -1 & 3 & -2 & 11 & 11,00 & 10,75 & \multicolumn{ 1}{c|}{} \\ \hline
-1 & 1 & 1 & 3 & 2 & 11 & 11,00 & 12,75 & \multicolumn{ 1}{c|}{10} \\ \cline{ 1- 8}
1 & -1 & 1 & 3 & 2 & 11 & 11,00 & 12,75 & \multicolumn{ 1}{c|}{} \\ \cline{ 1- 8}
1 & 1 & -1 & 3 & 2 & 11 & 11,00 & 12,75 & \multicolumn{ 1}{c|}{} \\ \hline
\end{tabular}
\end{center}
\label{tab1}
\end{table}
Die 4 untersten Bänder sind somit:
\begin{align*}
n = 1:\quad & \epsilon_{000}(x) = 3 x^2 \\
n = 2:\quad & \epsilon_{\bar1\bar1\bar1}(x) =  3 x^2 - 6 x + 3\\
n = 3:\quad & \epsilon_{\bar100}(x) = \epsilon_{0\bar10}(x) = \epsilon_{00\bar1}(x) =  3 x^2  - 2 x + 3\\
n = 4:\quad & \epsilon_{\bar1\bar10}(x) = \epsilon_{\bar10\bar1}(x) = \epsilon_{0\bar1\bar1}(x) = 3 x^2 - 4 x + 4\\
\end{align*}
\suppressfloats

\begin{wrapfigure}{R}{12cm}
  \centering
  \includegraphics[width=11.5cm]{aufgabe15b.png}
\label{bild15b}
% \caption{zu Aufgabe 15 b)}
\end{wrapfigure}
\subsubsection*{b)}
Überschneidungen gibt es zwischen der 1. und 2. Kurve bei $x = 0,5$, zwischen der
2. und 3. Kurve bei $x = 0$ und zwischen der 3. und 4. Kurve bei $x = 0,5$.
An diesen Stellen erwarten wir Aufspaltungen der Bänder.
\newpage
\begin{wrapfigure}{R}{12cm}
  \centering
  \includegraphics[width=11.5cm]{aufgabe15c.png}
\label{bild15c}
\caption{Bild 8.20 aus Gross, Marx: Festkörperphysik}
\end{wrapfigure}
\subsubsection*{c)}
\suppressfloats



\subsubsection*{d)}


\newpage
\subsection*{Aufgabe 16}
\subsubsection*{a)}
Im Modell quasifreier Elektronen wird angenommen, dass die Elektronen das Kernpotential als sehr schwach wahrnehmen, da sie sich in äußeren Schalen befinden und das Kernpotential weitesgehend abgeschirmt ist. Dadurch lassen sich die Zustände als freie Elektronen mit dem Kernpotential als kleine Störung beschreiben, was eine Darstellung mit ebenen Wellen $\mathrm{e}^{ikr}$ ermöglicht.\\
Im \textit{tight-binding} Modell werden sehr "tief" liegende Elektronen betrachtet, für die das Kernpotential dominant ist gegenüber anderen Wechselwirkungen. Eine Beschreibung mit ebenen Wellen ist daher nun nicht mehr möglich, sondern vielmehr müssen atomare Wellenfunktionen $\phi_A^i$ benutzt werden um die Elektronen zu beschreiben. Dabei wird häufig eine Linearkombinationen von atomren Wellenfunktionen angesetzt (LCAO \textit{Linear Combination of Atomic Orbitals}), da sich die Atomorbitale gegenseitig beeinflussen.\\
Versucht man die Schrödingergelichung für dieses Problem nun zu lösen, so kann man die sogenannte \textit{Einelektronen-Näherung} benutzen. Dabei wird das Potential des Atoms, welches dem Ort des Elektron am nächsten ist als dominantes Potential $V_A$ und die Potentiale aller anderen Atome in einem kleinen Störpotential $\tilde{V_A}$ zusammengefasst.\\

\subsubsection*{b)}
Das Austauschintegral: $ \beta_i = \int \mathrm{d}V \phi_A^i(r-R')\tilde{V}(r-R)\phi_A^i(r-R)$ führt in der Energieeigenwertgleichung $E(k) = E_A^i -\alpha^i-\beta^i\sum \mathrm{e}^{ik(R-R')}$ zu einer $k$-abhängigkeit der Energie und somit zu eiener endlichen breite des Bandes, die nur von dem Wert Austauschintegral abhängt.
\subsubsection*{c)}
Für einen eindimensionalen Kristall im Modell stark gebundener Elektronen ist mit $ka<<1$:
\begin{align*}
&\sum \mathrm{e}^{ik(R-R')} = 2\cos{2a}
\Rightarrow E(k) &= E_A^i -\alpha^i-\beta^i\cdot2\cos{2a} \approx E_A^i -\alpha^i-\beta^i\cdot2\left(1-\frac{(ka)^2}{2}\right)\\
\Rightarrow &= E_A^i -\alpha^i-\beta^i\cdot2-\beta^i(ka)^2
\end{align*}

Der Vergleich mit dem freien Elektronengas ($E(k) = \frac{\hbar^2 k^2}{m_{\mathrm{eff}}}$ und $\frac{1}{\hbar}\frac{\delta^2 E}{\delta k^2} = \frac{1}{m_{\mathrm{eff}}}$ liefert für die Effektive Masse:
\begin{align*}
\frac{1}{m_{\mathrm{eff}}} &= \frac{1}{\hbar}\frac{\delta^2 E}{\delta k^2} \\
 &= \frac{1}{\hbar} 2\beta^ia^2
\end{align*}
Die effektive Masse ist also gerade umgekehrt proportional zum Wert des Austauschintegrals $\beta^i$

\end{document}

