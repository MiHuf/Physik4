\subsection*{Aufgabe 24}

\subsubsection*{a)}
Magnetisierung $\vec M(\vec r, T)$, Magnetfeld $\vec H(\vec r, T)$, und Flußdichte
$\vec B(\vec r, T)$ hängen mit der magnetischen Feldkonstante $\mu_0$ wie folgt zusammen:
\begin{align*}
  \vec B &= \mu_0 (\vec H + \vec M) = \mu_r \mu_0 \vec H\; ;\qquad \vec M = \chi \cdot \vec H =  (\mu_r - 1) \vec H\\
  [B] &= \text{T} = \frac{\text{V s}}{\text{m}^2} \\
  [H] &= [M] = \frac{\text{A}}{\text{m}} \\
  [\mu_0] &= \frac{\text{V s}}{\text{A m}} \\
  [\epsilon_0] &= \frac{\text{A s}}{\text{V m}} \\
  [\vec \mu] = [\vec m] &= \frac{\text{J}}{\text{T}} = \text{A m}^2 \qquad \text{magnetisches Moment}\\
  \mu_B &=  \frac{e \hbar}{2 m_e} = 9{,}274 \cdot 10^{-24}\frac{\text{J}}{\text{T}}
    = 9{,}274 \cdot 10^{-24}\text{A m}^2  \qquad \text{Bohrsches Magneton}
\end{align*}

\subsubsection*{b)}

\subsubsection*{c)}

\subsubsection*{d)}

\subsubsection*{e)}

\subsubsection*{f)}

\subsubsection*{g)}
