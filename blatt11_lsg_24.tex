\subsection*{Aufgabe 24}

\subsubsection*{a)}
Hier auch gleich unsere Formelsammlung für den Spickzettel.
Magnetisierung $\vec M(\vec r, t)$, Magnetfeld $\vec H(\vec r, t)$, und Flußdichte
$\vec B(\vec r, t)$ hängen mit der magnetischen Feldkonstante $\mu_0$ wie folgt zusammen
($n$ ist die Teilchenzahldichte, $F$ = freie Energie):
\begin{align*}
  [\vec m] = [\vec \mu] &= \frac{\text{J}}{\text{T}} = \text{A m}^2 \qquad \text{magnetisches Moment}\\
  \vec M & = \frac{\vec m}{V} = - n \cdot \left(\frac{\partial F}{\partial B} \right)
  \qquad \text{Magnetisierung = magn. Moment pro Volumen}\\
  [M] = [H] &= \frac{\text{A}}{\text{m}} \\
  \vec M &= \chi \cdot \vec H =  (\mu_r - 1) \vec H
    \;;\quad \vec B = \mu_0 (\vec H + \vec M) = \mu_r \mu_0 \vec H \\
  \chi & = \mu_r - 1 = \mu_0 \left(\frac{\partial M}{\partial B_{ext}} \right)
  = - \frac{\mu_0}{V}  \left(\frac{\partial^2 F}{\partial B_{ext}^2} \right)_{T, V}
  \qquad \text{magnetische Suszeptibilität}\\
  [B] &= \text{T} = \frac{\text{V s}}{\text{m}^2} \\
  [\mu_0] &= \frac{\text{V s}}{\text{A m}} \\
  [\epsilon_0] &= \frac{\text{A s}}{\text{V m}} \\
  \mu_B &=  \frac{e \hbar}{2 m_e} = 9{,}274 \cdot 10^{-24}\frac{\text{J}}{\text{T}}
    = 9{,}274 \cdot 10^{-24}\text{A m}^2  \qquad \text{Bohrsches Magneton}
\end{align*}
Fehler im Skript bei der Formal für $\chi$.

\subsubsection*{b)}
Fehler im Aufgabenzettel bei der Formel für $<\mu>$.
\begin{align*}
<\mu> &= \mu_B \cdot \frac{\exp\left(+\frac{\mu_B B}{k_B T}\right) -
 \exp\left(-\frac{\mu_B B}{k_B T}\right)}
  {\exp\left(+\frac{\mu_B B}{k_B T}\right) + \exp\left(-\frac{\mu_B B}{k_B T}\right) } =
  \mu_B \cdot \frac{\sinh\left(\frac{\mu_B B}{k_B T}\right)}{\cosh\left(\frac{\mu_B B}{k_B T}\right)}\\
\intertext{Ein fcc-Gitter enthält  4 Atome pro Elementarzelle (Volumen $= a^3$), also:}
<M> &= \frac{4 \mu_B}{a^3}\cdot \tanh \left(\frac{\mu_B B}{k_B T}\right)
\intertext{Im Grenzfall $T \rightarrow 0$ ergibt das}
<M_S> &= \frac{4 \mu_B}{a^3} = \frac{4 \cdot 9{,}274 \cdot 10^{-24}\text{A m}^2}{8 \cdot 10^{-27} \text{m}^3}
 = 4637 \frac{\text{A}}{\text{m}}
\end{align*}

\subsubsection*{c)}
$J = \frac{1}{2}, L = 0$. Welche Seite im Skript?

\subsubsection*{d)}
\begin{align*}
  \frac{\partial}{\partial x} \tanh(x) &= \frac{1}{\cosh^2 (x)} \\
  \Rightarrow \quad \chi &= \mu_0 \left(\frac{\partial M}{\partial B}\right) =
  \frac{4 \mu_B^2}{a^3 k_B T} \cdot \frac{1}{\cosh^2\left(\frac{\mu_B B}{k_B T}\right)}
\end{align*}

\subsubsection*{e)}
Für hohe Temperaturen $T$ und kleine Magnetfelder $B$ ist $\frac{\mu_B B}{k_B T} \ll 1$
und ich kann in der Formal für $M$ den $\tanh(x)$ entwickeln $\tanh(x) \approx x - \frac{1}{3}x^3$:
\begin{align*}
  M &\approx \frac{4 \mu_B}{a^3} \cdot \frac{\mu_B B}{k_B T}\\
  \chi &\approx \frac{4 \mu_B^2}{a^3 k_B T} \propto \frac{1}{T}
\end{align*}



\subsubsection*{f)}

\subsubsection*{g)}
