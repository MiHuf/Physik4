% DOCUMENT CLASS
\documentclass[11pt]{article}
%PACKAGES
\usepackage[utf8]{inputenc}
\usepackage[ngerman]{babel}
\usepackage[reqno,fleqn]{amsmath}
\setlength\mathindent{10mm}
\usepackage{amstext}
\usepackage{amssymb}
\usepackage{fancyhdr}
\usepackage{units}
\usepackage{comment}  % mehrzeilige Kommentare

% Grafik
\usepackage{graphicx}
\usepackage{subfigure}
\usepackage{wrapfig}
% Zahlenwerte mit Einheiten mittels \unit{Zahlenwert}{Einheit}
\usepackage[thinspace,thinqspace,squaren,textstyle]{SIunits}
% FORMATIERUNG
\usepackage[paper=a4paper,left=25mm,right=25mm,top=25mm,bottom=25mm]{geometry}
\setlength{\parindent}{0cm}
\setlength{\parskip}{1.5mm plus1mm minus1.5mm}
% PAGESTYLE
\pagestyle{fancy}
\setlength\headheight{30pt}
\lhead{Michael Hufschmidt, Mat. Nr. 6436122\\Florian Jochheim, Mat. Nr. 6508131}
\rhead{Übungen zur Physik IV, SoSe 2015\\Blatt 09 zum 22.06.2015}
%MATH SHORTCUTS
\newcommand*{\NN}{\mathbb N}
\newcommand*{\ZZ}{\mathbb Z}
\begin{document}
\begin{center}
\begin{tabular}{|l|c|c|c|c|}
\hline
Aufgabe:\quad &\quad 20 \quad&\quad 21 \quad&\quad \quad&\quad $\sum$ \quad\\
\hline
mögliche Punkte: \quad& 5 & 5 &  & 10 \\
\hline
erreichte Punkte: &  &  &  & \\
\hline
\end{tabular}
\end{center}

\subsection*{Aufgabe 20}

\subsubsection*{a)}
Gegeben ist die parabolische Dispersionsrelation für ein Material:
\begin{align*}
  E(\vec k) = \frac{\hbar ^2}{2} \left( \frac{k_x^2}{m_x} + \frac{k_y^2}{m_y} + \frac{k_z^2}{m_z}\right)
\end{align*}
Die Fermi-Fläche ist also ein Rotationsellipsoid.
Wenn sich das Material, in einem Magnetfeld in $z$-Richtung befindet, bewegen sich die
Elektronen auf Orbits in der in der $x$-$y$-Ebene auf entsprechenden
Schnitten durch das Rotationsparaboloid. Algebraische Umformungen liefern:
\begin{align*}
  &E - \frac{\hbar^2 k_z^2}{2 m_z} = \left( \frac{\hbar ^2 k_x^2}{2 m_x} + \frac{\hbar ^2 k_y^2}{2 m_y} \right)\\
  & 1 = \left( \frac{\hbar ^2 k_x^2}{2 m_x \cdot \left(E - \frac{\hbar^2 k_z^2}{2 m_z} \right)} +
    \frac{\hbar ^2 k_y^2}{2 m_y \cdot \left(E - \frac{\hbar^2 k_z^2}{2 m_z} \right)} \right)
\end{align*}
Das ist die Gleichung einer Ellipse $1 = \frac{k_x^2}{a^2} + \frac{k_y^2}{b^2}$ mit den Halbachsen
\begin{align*}
  a = \sqrt{\frac{2 m_x}{\hbar^2} \cdot \left( E - \frac{\hbar^2 k_z^2}{2 m_z} \right) }\;; \qquad
  b = \sqrt{\frac{2 m_y}{\hbar^2} \cdot \left( E - \frac{\hbar^2 k_z^2}{2 m_z} \right) }
\end{align*}
Die Fläche einer Ellipse beträgt $A = \pi  \cdot a  \cdot b$, also:
\begin{align*}
A_k(E, k_z) &= \frac{2 \pi}{\hbar^2} \cdot \sqrt{m_x \cdot m_y}\left( E - \frac{\hbar^2 k_z^2}{2 m_z} \right)
\intertext{Damit wird:}
\frac{\partial A_k}{\partial E} &= \frac{2 \pi}{\hbar^2} \cdot \sqrt{m_x \cdot m_y}\\
T_C &= \frac{\hbar^2}{e B} \cdot \frac{\partial A_k}{\partial E} = \frac{2 \pi}{e B}  \cdot \sqrt{m_x \cdot m_y}
\end{align*}
Die Umlaufzeit ist also unabhängig von $k_z$.

\subsubsection*{b)}
Für die Zyklotronmasse $m^*$ gilt dann:
\begin{align*}
  m^* = \frac{\hbar^2}{2 \pi} \cdot \frac{\partial A_k}{\partial E} = \sqrt{m_x \cdot m_y}
\end{align*}

\subsubsection*{c)}
Gegeben sei ein Magnetfeld $\vec B = B \cdot \begin{pmatrix} n_x \\n_y \\n_z \end{pmatrix}$,
und die Bewegungsgleichung für ein Elektron
\begin{align*}
\begin{pmatrix} m_x&0&0 \\0&m_y&0 \\0&0&m_z\end{pmatrix} \frac{\partial}{\partial t} \vec v=
 q \, \vec v \times \vec B = q B \begin{pmatrix}n_z v_y - n_y v_z \\ n_x v_z - n_z v_x \\ n_y v x - n_x v_y\end{pmatrix}
\end{align*}
Mit dem Ansatz $\vec v = \vec v_0 \mathrm e^{i \omega t}$ ergibt sich
$\frac{\partial}{\partial t} \vec v = i \omega \vec v$ sowie:
\begin{align*}
\begin{pmatrix}\frac{i \omega m_x}{q B}&0&0\\0&\frac{i \omega m_y}{q B}&0\\0&0\frac{i \omega m_z}{q B}\\\end{pmatrix}
  \cdot \begin{pmatrix}v_x\\v_y \\ v_z\end{pmatrix} =\begin{pmatrix}n_z v_y - n_y v_z \\ n_x v_z - n_z v_x \\ n_y v x - n_x v_y\end{pmatrix}
\end{align*}



% \newpage
\subsection*{Aufgabe 21}

\subsubsection*{a)}
In Aufgabe 15 d) haben wir für die Fermi-Energie eines fcc-Kristalls mit der
Gitterkonstanten $a$ bestehend aus Atomen der Wertigkeit $Z$ im Modell
quasifreier Elektronen den folgenden Ausdruck hergeleitet:
\begin{align*}
E_F &=  \frac{\hbar^2}{2 m_e} \cdot k_F^2 =   \frac{\hbar^2}{2 m_e} \cdot
  \left(\frac{3 Z}{2 \pi} \right)^{2/3} \cdot \left(\frac{2 \pi}{a} \right)^2
\end{align*}


\subsubsection*{b)}

\subsubsection*{c)}


% \newpage

\end{document}

