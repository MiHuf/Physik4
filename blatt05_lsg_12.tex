\subsection*{Aufgabe 12}
\subsubsection*{a)}
Sei $A$ die ausgeleuchtete Fläche auf der Oberfläche des Elektronengases. Die Energie die auf das Elektronengas übertragen würde ergibt sich aus:
\begin{align*}
\delta E = F\cdot A
\end{align*}
Es gilt:
\begin{align*}
\Delta E &= \Delta Q= c_{\nu,el}\cdot V\cdot\Delta T\\
F\cdot A&= c_{\nu,el}\cdot V\cdot\Delta T\\
\rightarrow \Delta T &= \frac{F\cdot A}{c_{\nu,el}\cdot V} = \frac{F\cdot A}{c_{\nu,el}\cdot A\cdot \unit{20}{nm}} =  \frac{\unit{140}{Jm^{-2}}}{\unit{400}{Jm^{-3}K^{-2}}\cdot \unit{20}{nm}\cdot\unit{10}{K}} =8,75\cdot10^3K\\
T&=T_0+\Delta T = (10+8,75\cdot10^3)K = 87603K
\end{align*}

\subsubsection*{b)}
Analog zu a) ist:
\begin{align*}
\Delta E &= \Delta Q= c_{\nu,ph}\cdot V\cdot\Delta T\\
F\cdot A&= c_{\nu,ph}\cdot V\cdot\Delta T\\
\rightarrow \Delta T &= \frac{F\cdot A}{c_{\nu,ph}\cdot V} = \frac{F\cdot A}{c_{\nu,ph}\cdot A\cdot \unit{20}{nm}} \\
&=\frac{\unit{140}{Jm^{-2}}}{3 \cdot 7,4\cdot10^{28}\unit{}{m^{-3}}\cdot1,381\cdot10^{-23}\unit{}{JK^{-1}} \unit{20}{nm}}=
\unit{2,283}{\cdot 10^3}{\kelvin}\\
T&=T_0+\Delta T = (10+2283)K = 2293 K
\end{align*}
\subsubsection*{c)}
Die durch das Elektronengas aufgenommene Wärmemenge sei $\Delta Q_e$ und die durch Gitter aufgenommene Wärmemenge $\Delta Q_k$. Dann muss gelten:
\begin{align*}
\Delta Q_e+\Delta Q_k &= \Delta E = \delta E = F\cdot A\\
\rightarrow F\cdot A &=c_{\nu,ph}\cdot V\cdot\Delta T+c_{\nu,el}\cdot V\cdot\Delta T = (c_{\nu,ph}+c_{\nu,el})\cdot V\cdot\Delta T\\
\Delta T &= \frac{F\cdot A}{(c_{\nu,ph}+c_{\nu,el})\cdot V}\\
&= \frac{\unit{140}{Jm^{-2}}\cdot A}{(3 \cdot 7,4\cdot10^{28}\unit{}{m^{-3}}\cdot1,381\cdot10^{-23}\unit{}{JK^{-1}}+\unit{400}{Jm^{-3}K^{-2}}\cdot\unit{10}{K})\cdot A\cdot \unit{20}{nm}}\\
&= 2,280\cdot10^3\kelvin\\
\rightarrow T &= 2290 K
\end{align*}
