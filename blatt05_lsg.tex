% DOCUMENT CLASS
\documentclass[11pt]{article}
%PACKAGES
\usepackage[utf8]{inputenc}
\usepackage[ngerman]{babel}
\usepackage[reqno,fleqn]{amsmath}
\setlength\mathindent{10mm}
\usepackage{amstext}
\usepackage{amssymb}
\usepackage{fancyhdr}
% Grafik
\usepackage{graphicx}
\usepackage{subfigure}
\usepackage{wrapfig}
% Zahlenwerte mit Einheiten mittels \unit{Zahlenwert}{Einheit}
\usepackage[thinspace,thinqspace,squaren,textstyle]{SIunits}
% FORMATIERUNG
\usepackage[paper=a4paper,left=25mm,right=25mm,top=25mm,bottom=25mm]{geometry}
\setlength{\parindent}{0cm}
\setlength{\parskip}{1.5mm plus1mm minus1.5mm}
% PAGESTYLE
\pagestyle{fancy}
\setlength\headheight{30pt}
\lhead{Michael Hufschmidt, Mat. Nr. 6436122\\Florian Jochheim, Mat. Nr. 6508131}
\rhead{Übungen zur Physik IV, SoSe 2015\\Blatt 05 zum 11.05.2015}
%MATH SHORTCUTS
\newcommand*{\NN}{\mathbb N}

\begin{document}

\subsection*{Aufgabe 10}
\subsubsection*{a)}
Die Phononen haben nach resonanter Anregung ebenfalls eine Frequenz von
$f =$ 200 MHz, die Energie eines einzelnen Phonons ist dann $E = h f =
\unit{6,626}{\cdot 10^{-34}}{\joule\second} \cdot \unit{200}{\mega\hertz} =
\unit{1,325}{\cdot 10^{-25}}{\joule}$. Wenn der Ultraschallimpuls reflexionsfrei
auf der Fläche $A = \unit{1}{\centi\meter\squared}$ eingekoppelt wird, wird die
Energie $E = I \cdot A \cdot t = \unit{1}{\milli\watt\per\centi\meter\squared}
\cdot \unit{1}{\centi\meter\squared} \cdot \unit{10}{\micro\second} =
\unit{1}{\cdot 10^{-8}}{\joule}$ auf den Kristall übertragen. Damit werden dann
$7,546 \cdot 10^{16}$ Phononen erzeugt.
\subsubsection*{b)}
Da die Temperatur $T = \unit{4,2}{\kelvin}$ klein gegen die Debye-Temperatur
$\varTheta_D = \unit{640}{\kelvin}$ ist, können wir mit akkustischen Phononen im
Debye-Modell rechnen. Für $T \ll \varTheta_D$ gilt:
\begin{align*}
  c_V = 3 k_B N \cdot \frac{4 \pi^4}{5} \left(\frac{T}{\varTheta_D}\right)^3
\end{align*}
Die würfelförmige Elementarzelle des Diamantgitters von Silizium (Volumen
$a^3 = (\unit{0,543}{\pico\meter})^3$) enthält 8 Atome (vgl. Übungsaufgabe 1.c) ),
damit enthält der Kristall $N = xxx$ Atome.


\subsubsection*{c)}


\subsection*{Aufgabe 11}
Es ist im allgemeinen Fall:
\begin{align*}
D(E)=\frac{1}{\Delta E}\int_{k(E)}^{k(E+\Delta E)} Z(k) d^3k\end{align*} 
wobei sich das Volumenintegral im k-Raum abhängig von der Dimension vereinfacht:
\begin{align*}
Z_1(k) &= \frac{L}{2\pi} \rightarrow \int_{\frac{E}{c}}^{\frac{E+\Delta E}{c}} Z_1(k) dk = \frac{L}{2\pi} \frac{\Delta E}{c} \rightarrow D(E)=\frac{L}{2c\pi}\\
Z_2(k) &= \frac{A}{(2\pi)^2} \rightarrow \int_{\frac{E}{c}}^{frac{E+\Delta E}{c}} Z_2(k) d^2k = \frac{A}{(2\pi)^2} 2\pi k \Delta k =\frac{A}{2\pi} \frac{k}{c} \Delta E = \frac{A}{2\pi}\frac{E}{c^2} \Delta E \\
&\rightarrow D_2(E) = \frac{A}{2\pi}\frac{E}{c^2}\\
Z_3(k) &= \frac{V}{(2\pi)^3} \rightarrow \int_{\frac{E}{c}}^{frac{E+\Delta E}{c}} Z_3(k) d^3k = \frac{V}{(2\pi)^3}\int_{\frac{E}{c}}^{frac{E+\Delta E}{c}} d^3k = \frac{V}{(2\pi)^3}4\pi k^2 \Delta k = 2\frac{V}{(2\pi)^2} \frac{E^2}{c^3} \Delta E\\
&\rightarrow D_3(E) = 2\frac{V}{(2\pi)^2} \frac{E^2}{c^3}
\end{align*}

\subsection*{Aufgabe 12}
\subsubsection*{a)}
Sei $A$ die ausgeleuchtete Fläche auf der Oberfläche des Elektronengases. Die Energie die auf das Elektronengas übertragen würde ergibt sich aus:
\begin{align*}
\delta E = F\cdot A
\end{align*}
Es gilt:
\begin{align*}
\Delta E &= \Delta Q= c_{\nu,el}\cdot V\cdot\Delta T\\
F\cdot A&= c_{\nu,el}\cdot V\cdot\Delta T\\
\rightarrow \Delta T &= \frac{F\cdot A}{c_{\nu,el}\cdot V} = \frac{F\cdot A}{c_{\nu,el}\cdot A\cdot \unit{20}{nm}} =  \frac{\unit{140}{Jm^{-2}}}{\unit{400}{Jm^{-3}K^{-2}}\cdot \unit{20}{nm}\cdot\unit{10}{K}} =8,75\cdot10^3K\\
T&=T_0+\Delta T = (10+8,75\cdot10^3)K = 87603K
\end{align*}

\subsubsection*{b)}
Analog zu a) ist:
\begin{align*}
\Delta E &= \Delta Q= c_{\nu,ph}\cdot V\cdot\Delta T\\
F\cdot A&= c_{\nu,ph}\cdot V\cdot\Delta T\\
\rightarrow \Delta T &= \frac{F\cdot A}{c_{\nu,ph}\cdot V} = \frac{F\cdot A}{c_{\nu,ph}\cdot A\cdot \unit{20}{nm}} \\
&=\frac{\unit{140}{Jm^{-2}}}{3 \cdot 7,4\cdot10^{28}\unit{}{m^{-3}}\cdot1,381\cdot10^{-23}\unit{}{JK^{-1}} \unit{20}{nm}}=
\unit{2,283}{\cdot 10^3}{\kelvin}\\
T&=T_0+\Delta T = (10+2283)K = 2293 K
\end{align*}
\subsubsection*{c)}
Die durch das Elektronengas aufgenommene Wärmemenge sei $\Delta Q_e$ und die durch Gitter aufgenommene Wärmemenge $\Delta Q_k$. Dann muss gelten:
\begin{align*}
\Delta Q_e+\Delta Q_k &= \Delta E = \delta E = F\cdot A\\
\rightarrow F\cdot A &=c_{\nu,ph}\cdot V\cdot\Delta T+c_{\nu,el}\cdot V\cdot\Delta T = (c_{\nu,ph}+c_{\nu,el})\cdot V\cdot\Delta T\\
\Delta T &= \frac{F\cdot A}{(c_{\nu,ph}+c_{\nu,el})\cdot V}\\
&= \frac{\unit{140}{Jm^{-2}}\cdot A}{(3 \cdot 7,4\cdot10^{28}\unit{}{m^{-3}}\cdot1,381\cdot10^{-23}\unit{}{JK^{-1}}+\unit{400}{Jm^{-3}K^{-2}}\cdot\unit{10}{K})\cdot A\cdot \unit{20}{nm}}\\
&= 2,280\cdot10^3\kelvin\\
\rightarrow T &= 2290 K
\end{align*}


\end{document}

