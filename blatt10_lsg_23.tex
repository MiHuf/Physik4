\subsection*{Aufgabe 23}

\subsubsection*{a)}
Bei einem zweidimensionalen Elektronensystem befinden sich die Elektronen auf Energieniveaus $\lambda =0,1,2...$, die die Form von Kreisen haben mit der Energie 
\begin{align*}
 E = \hbar\omega_c\left(\lambda+\frac 12\right)
\end{align*} 
wobei $\omega_c$ die Zyklotronfrequenz ist. Die Entartung eines Niveaus ist dabei gerade $2N_L = 2\frac{eB}{h}$ Elektronen.

Bei N Elektronen sind die ersten s Landau Niveaus, also bis zum Niveau s-1 komplett aufgefüllt und das Niveau s+1-te Niveau, also das s Niveau ist nur Teilweise gefülllt.
Die gesamt Energie der vollständigen Niveaus ist dann:

\begin{align}
 E_{tot}^s &= \sum_{n=0}^{s-1} 2N_L\hbar \omega_c\left(n+ \frac 12\right) 
 \intertext{Wir machen einen Indexshift und ziehen die Summen auseinander}
 &= 2N_L\hbar\omega_c\left(\sum_{n=1}^s n + \sum_{n=1}^s \frac 12\right)\\
 &= 2N_L\hbar\omega_c\left(\frac{s(s+1)}{2} + \frac 12 s\right) = 2N_L\omega_c = N_L\hbar\omega_c s^2
\end{align}
Das teilweise gefüllte s+1-te Niveau ist mit $N-2sN_L$ Elektronen besetzt und hat somit die Energie:
\begin{align*}
E_{tot}^{s+1} = \hbar\omega_c\left(s+\frac 12\right)(N-2sN_L)
\end{align*}
Somit ergibt sich für die gesammte freie Energiedichte $U$:
\begin{align*}
U &=  \frac 1V \left(N_L\hbar\omega_c s^2+\hbar\omega_c\left(s+\frac 12\right)(N-2sN_L)\right)\\
\intertext{Mit $\frac{N_L}{V}:= n_L$ und $\frac{N}{V} = n$ ergibt sich:}
& = \hbar\omega_c\left(n_L s^2+\left(s+\frac 12\right)(n-2sn_L)\right)\\
&= \hbar\omega_c\left(n_L s^2+\left(s+\frac 12\right)(-2sn_L)+n\left(s+\frac 12\right)\right)\\
&= \hbar\omega_c\left(n_L s^2-2n_L s^2-sn_L+n\left(s+\frac 12\right)\right)\\
&= \hbar\omega_c\left(-n_L s^2-sn_L+n\left(s+\frac 12\right)\right)\\
&= \hbar\omega_c\left(n\left(s+\frac 12\right)-n_L s(s+1)\right)\\
\end{align*}
\subsubsection*{b)}
Mit $n_L = 2\frac{eB}{h}$ und $\omega_c = \frac{eB}{m^*}$ können wir die Magnetisierung M berechnen:
\begin{align*}
M &= \frac{\delta U}{\delta B}\\
 &= \frac{\delta}{\delta B} \left[\hbar\omega_c\left(n\left(s+\frac 12\right)-n_L s(s+1)\right)\right] \\
 &= \frac{\delta}{\delta B} \left[\hbar\frac{eB}{m^*}\left(n\left(s+\frac 12\right)-\frac{eB}{h} s(s+1)\right)\right]\\
 &= \hbar\left(\frac{en}{m^*}\left(s+\frac 12\right)-\frac{e^2B}{hm^*} s(s+1)\right)
\end{align*}

\subsubsection*{c)}
Die Magnetiesierung ist offensichtlich linear in B. Setzen wir $B = \frac{hn}{2e(s+1)}$ so erhalten wir:
\begin{align*}
M &= \hbar\left(\frac{en}{m^*}\left(s+\frac 12\right)-\frac{e^2B}{hm^*} s(s+1)\right)\\
M &= \hbar\left(\frac{en}{m^*}\left(s+\frac 12\right)-\frac{e^2\frac{hn}{2e(s+1)}}{hm^*} s(s+1)\right) \\
&= \hbar\left(\frac{en}{m^*}\left(s+\frac 12\right) - \frac{en}{m^*}s+\frac{en}{m^*}\right)\\
&= -\frac{\hbar en}{2m^*} = -n \mu_B
\end{align*}

Obere Grenze fehlt noch...
