% DOCUMENT CLASS
\documentclass[11pt]{article}
%PACKAGES
\usepackage[utf8]{inputenc}
\usepackage[ngerman]{babel}
\usepackage[reqno,fleqn]{amsmath}
\setlength\mathindent{10mm}
\usepackage{amstext}
\usepackage{amssymb}
\usepackage{fancyhdr}
\usepackage{units}
\usepackage{comment}  % mehrzeilige Kommentare
\usepackage{cancel} %Brueche kuerzen
\usepackage[table]{xcolor}
\usepackage{tabularx}
\newcolumntype{L}[1]{>{\raggedright\arraybackslash}p{#1}} % linksbündig mit Breitenangabe
\newcolumntype{C}[1]{>{\centering\arraybackslash}p{#1}} % zentriert mit Breiten
% Grafik
\usepackage{graphicx}
\usepackage{subfigure}
\usepackage{wrapfig}
% Zahlenwerte mit Einheiten mittels \unit{Zahlenwert}{Einheit}
\usepackage[thinspace,thinqspace,squaren,textstyle]{SIunits}
% FORMATIERUNG
\usepackage[paper=a4paper,left=25mm,right=25mm,top=25mm,bottom=25mm]{geometry}
\setlength{\parindent}{0cm}
\setlength{\parskip}{1.5mm plus1mm minus1.5mm}
% PAGESTYLE
\pagestyle{fancy}
\setlength\headheight{30pt}
\lhead{Michael Hufschmidt, Mat. Nr. 6436122\\Florian Jochheim, Mat. Nr. 6508131}
\rhead{Übungen zur Physik IV, SoSe 2015\\Blatt 10 zum 29.06.2015}
%MATH SHORTCUTS
\newcommand*{\NN}{\mathbb N}
\newcommand*{\ZZ}{\mathbb Z}
\begin{document}
\begin{center}
\begin{tabular}{|L{3cm}|C{1cm}|C{1cm}|C{1cm}|}
\hline
Aufgabe:&\cellcolor{blue!25}22 &\cellcolor{blue!25} 23 &\cellcolor{blue!25} $\sum$\\
\hline
mögliche Punkte: & 5 & 5 & 10 \\
\hline
erreichte Punkte: &  &  & \\
\hline
\end{tabular}
\end{center}

\subsection*{Aufgabe 22}

\subsubsection*{a)}
\begin{itemize}
  \item {n-Bereich $x>0$}:
Hier ist $N_A = 0$ und für die Dichte der Elektronen gilt in der Boltzmann-Näherung:
\begin{align}
\nonumber
& n_c(T) = N_c^{eff}(T)\cdot e^{- \frac{E_c - \mu_n}{k_B T}}
\intertext{Bei vollständiger Ionisierung der Störstellen gilt $n_c(T) = N_D$,
  Umformungen liefern dann:}
\nonumber
& k_B T \cdot \ln \left(\frac{N_D}{N_c^{eff}}\right) = - (E_c - \mu_n)\; ; \quad \Rightarrow \quad
  \mu_n = E_c - k_B T \cdot \ln \left(\frac{N_c^{eff}}{N_D}\right)
\intertext{Bei Raumtemperatur $T = 20^\circ$C = 293,15\,K ist $k_B T =$ 25,26\,meV,
mit den Zahlenwerten aus dem Aufgabenblatt ergibt sich dann:}
\label{eq-un}
&\mu_n = E_c - \text{25,26\,meV} \cdot \ln \left(\frac{2{,}8\cdot10^{19}}{1 \cdot10^{17}}\right)
= E_c - \text{142,33\,meV}
\end{align}
  \item {p-Bereich $x<0$}: Analog ist hier $N_D = 0$ und für die Löcherdichte gilt:
\begin{align}
\nonumber
& p_v(T) = P_v^{eff}(T)\cdot e^{- \frac{\mu_p - E_v}{k_B T}} = N_A\\
\nonumber
&\quad \Rightarrow \quad k_B T \cdot \ln \left(\frac{N_A}{P_v^{eff}}\right) = E_v - \mu_p
\; ; \quad \Rightarrow \quad \mu_p = E_v + k_B T \cdot \ln \left(\frac{P_v^{eff}}{N_A}\right)
\intertext{Mit Zahlenwerten:}
\label{eq-up}
&\mu_p = E_v + \text{25,26\,meV} \cdot \ln \left(\frac{1{,}0\cdot10^{19}}{5\cdot10^{16}}\right)
= E_v + \text{133,84\,meV}
\end{align}
\end{itemize}
Bei Störstellenerschöpfung sind die Ladungsträgerdichten und somit die chemischen
Potentiale unabhängig von der Akzeptor-- bzw. Donator--Energie.
Aus \eqref{eq-un} und \eqref{eq-up} ergibt sich für die Diffusionspannung als
Abstand der beiden chemischen Potentiale:
\begin{align*}
V_D = \mu_n - \mu_p = E_c - E_v - \text{276,17\,meV} =
   \text{1,12\,eV} - \text{276,17\,meV} = \text{0,84\,eV}
\end{align*}

\subsubsection*{b)}

\subsubsection*{c)}

% \newpage
\subsection*{Aufgabe 23}

\subsubsection*{a)}
Bei einem zweidimensionalen Elektronensystem befinden sich die Elektronen auf Energieniveaus $\lambda =0,1,2...$, die die Form von Kreisen haben mit der Energie
\begin{align*}
 E = \hbar\omega_c\left(\lambda+\frac 12\right)
\end{align*}
wobei $\omega_c$ die Zyklotronfrequenz ist. Die Entartung eines Niveaus ist dabei gerade $2N_L = 2\frac{eB}{h}$ Elektronen.

Bei N Elektronen seien die ersten $s$ Landau Niveaus, also von 0 bis zum Niveau $s-1$ komplett aufgefüllt und das  $s+1$-te Niveau, also das Niveau $s$ sei nur teilweise gefülllt.
Die gesamt Energie der vollständige gefüllten  Niveaus ist dann:
\begin{align}
 E_{tot}^s &= \sum_{n=0}^{s-1} 2N_L\hbar \omega_c\left(n+ \frac 12\right)
 \intertext{Wir machen einen Indexshift und ziehen die Summen auseinander}
 &= 2N_L\hbar\omega_c\left(\sum_{n=1}^s n - \sum_{n=1}^s \frac 12\right)\\
 &= 2N_L\hbar\omega_c\left(\frac{s(s+1)}{2} - \frac 12 s\right) = 2N_L\omega_c = N_L\hbar\omega_c s^2
\end{align}
Das teilweise gefüllte $s+1$-te Niveau ist mit $N-2sN_L$ Elektronen besetzt und hat somit die Energie:
\begin{align*}
E_{tot}^{s+1} = \hbar\omega_c\left(s+\frac 12\right)(N-2sN_L)
\end{align*}
Somit ergibt sich für die gesammte freie Energiedichte $U$:
\begin{align*}
U &=  \frac 1V \left(N_L\hbar\omega_c s^2+\hbar\omega_c\left(s+\frac 12\right)(N-2sN_L)\right)\\
\intertext{Mit $\frac{N_L}{V} =: n_L$ und $\frac{N}{V} =: n$ ergibt sich:}
& = \hbar\omega_c\left(n_L s^2+\left(s+\frac 12\right)(n-2sn_L)\right)\\
&= \hbar\omega_c\left(n_L s^2+\left(s+\frac 12\right)(-2sn_L)+n\left(s+\frac 12\right)\right)\\
&= \hbar\omega_c\left(n_L s^2-2n_L s^2-sn_L+n\left(s+\frac 12\right)\right)\\
&= \hbar\omega_c\left(-n_L s^2-sn_L+n\left(s+\frac 12\right)\right)\\
&= \hbar\omega_c\left(n\left(s+\frac 12\right)-n_L s(s+1)\right)\\
\end{align*}
\subsubsection*{b)}
Mit $n_L = \frac{eB}{h}$ und $\omega_c = \frac{eB}{m^*}$ können wir die Magnetisierung $M$ berechnen:
\begin{align*}
M &= - \frac{\partial U}{\partial B}\\
 &= - \frac{\partial}{\partial B} \left[\hbar\omega_c\left(n\left(s+\frac 12\right)-n_L s(s+1)\right)\right] \\
 &= - \frac{\partial}{\partial B} \left[\hbar\frac{eB}{m^*}\left(n\left(s+\frac 12\right)-\frac{eB}{h} s(s+1)\right)\right]\\
 &= - \hbar\left(\frac{en}{m^*}\left(s+\frac 12\right)-\frac{2e^2B}{hm^*} s(s+1)\right)
\end{align*}

\subsubsection*{c)}
Die Magnetiesierung ist offensichtlich linear in $B$. Setzen wir $B = \frac{hn}{2e(s+1)}$ so erhalten wir:
\begin{align*}
M &= -\hbar\left(\frac{en}{m^*}\left(s+\frac 12\right)-\frac{2e^2B}{hm^*} s(s+1)\right)\\
M &= -\hbar\left(\frac{en}{m^*}\left(s+\frac 12\right)-\frac{2e^2\frac{hn}{2e(s+1)}}{hm^*} s(s+1)\right) \\
&= -\hbar\left(\frac{en}{m^*}\left(s+\frac 12\right) - \frac{en}{m^*}s\right)\\
&= -\frac{\hbar en}{2m^*} = -n \mu_B
\end{align*}

Und für $B = \frac{hn}{2es}$:
\begin{align*}
M &= -\hbar\left(\frac{en}{m^*}\left(s+\frac 12\right)-\frac{2e^2B}{hm^*} s(s+1)\right)\\
M &= -\hbar\left(\frac{en}{m^*}\left(s+\frac 12\right)-\frac{2e^2\frac{hn}{2e(s)}}{hm^*} s(s+1)\right) \\
&= -\hbar\left(\frac{en}{m^*}\left(s+\frac 12\right) - \frac{en}{m^*}(s+1)\right)\\
&= \frac{\hbar en}{2m^*} = n \mu_B
\end{align*}

Also steigt die Magnetisierung linear von $-n \mu_B$ auf $n \mu_B$ in dem gegebenem Bereich

% \newpage

\end{document}

