\subsection*{Aufgabe 17}
Gegeben ist die Dispersionsrelation für ein fcc-Gitter:
\begin{align}
\label{eq-disp1}
  E(k) = E'_\alpha - 4\;|A| \left(\cos\frac{k_x a}{2}\cos\frac{k_y a}{2} +
    \cos\frac{k_y a}{2}\cos\frac{k_z a}{2} + \cos\frac{k_z a}{2}\cos\frac{k_x a}{2} \right)
\end{align}

\subsubsection*{a)}
\begin{wrapfigure}[11]{R}{8cm}
  \centering
  \includegraphics[width=7.5cm]{aufgabe16b.png}
\label{bild16b}
\end{wrapfigure}
Im LCAO-Modell beschreibt das Austauschintegral A den Overlap der Wellenfunktion
eines (Valenz-)Elektrons mit den Wellenfunktionen der (Valenz-)Elektronen benachbarter
Atome. Daraus resultiert eine Aufspaltung der Bänder; diese ist propotional zu A
und einem Geometriefaktor (runde Klammer in obiger Formel) und ist dadurch
auch abhängig von $\vec k$.
\newline
\subsubsection*{b)}
\begin{itemize}
\item[$\Gamma$ X:]
  In $\Gamma$X-Richtung ist $k_y = k_z = 0$, sei $k:= k_x$, dann wird \eqref{eq-disp1}
  wegen $\cos(0) = 1$:
\begin{align*}
    E(k) = E'_\alpha - 4\;|A| \left(1 + 2 \cos\frac{k a}{2} \right)
\end{align*}
Siehe rote Kurve fX im Plot. Da $\cos(k)$ Werte zwischen -1 und +1 annehmen kann
ergibt sich eine Bandbreite $B = 16\;|A|$.
\item[$\Gamma$ L:]
  In $\Gamma$L-Richtung ist $k_x = k_y = k_z =: k$, dann wird \eqref{eq-disp1}:
\begin{align*}
    E(k) = E'_\alpha - 4\;|A| \left(3 \cos^2\frac{k a}{2} \right)
\end{align*}
Siehe grüne Kurve fL im Plot. Da $\cos^2(k)$ Werte zwischen 0 und +1 annehmen kann
ergibt sich eine Bandbreite $B = 12\;|A|$.
\end{itemize}

\subsubsection*{c)}
Die im Aufgabenblatt rot eingezeichneten Basisvektoren des zweidimensionalen
hexagonalen Gitters haben die Koordinaten
\begin{align}
\label{eq-a}
  \vec a_1 = a \cdot \begin{pmatrix} 1 \\ 0 \end{pmatrix} \;; \qquad
  \vec a_2 = a \cdot \begin{pmatrix} \cos(120^\circ)\\ \sin(120^\circ)\end{pmatrix} =
  a \cdot \begin{pmatrix} -\frac{1}{2}\\\frac{1}{2} \sqrt{3}\end{pmatrix}
\end{align}
\begin{comment}
(Zur Berechnung der inversen Gittervektoren ein kurzer Ausflug in drei Dimensionen.)
Damit wird $ | \vec a_1 \times \vec a_2 | = \frac{1}{2} \sqrt{3} a^2$ und für die
Basisvektoren im inversen Gitter ergibt sich:
\begin{align*}a^2}
\vec b_1 =  \frac{2 \pi}{\frac{1}{2} \sqrt{3} a^2} (\vec a_2 \times \vec e_3 ) =
  \frac{2 \pi}{a} \begin{pmatrix} 1 \\ 1 / \sqrt{3} \\ 0 \end{pmatrix} \;; \qquad
  \vec b_2 =  \frac{2 \pi}{\frac{1}{2} \sqrt{3} a^2} (\vec e_3 \times \vec a_1) =
  \frac{2 \pi}{a} \begin{pmatrix} 0 \\ 2 / \sqrt{3} \\ 0 \end{pmatrix}
\end{align*}
Man verifiziert leicht: $\vec a_i \cdot \vec b_j = 2 \pi \delta_{ij}$. Jetzt wieder
zweidimensional:
\end{comment}
Im Real-Raum hat ein Atom im Koordinatenursprung 6 nächste Nachbarn
mit den Koordinaten $R_n\;;\;n = 1, \cdots , 6$ im Gegen-Uhrzeigersinn:
\begin{align}
\label{eq-R}
\left\lbrace R_n\right\rbrace &= \left\lbrace
 \vec a_1,\; \vec a_1 + \vec a_2,\; \vec a_2, -\vec a_1,\; -\vec a_1 - \vec a_2,\; - \vec a_2
  \right\rbrace
\intertext{Für die Dispersion gilt im LCAO-Modell mit der "`nächste-Nachbarn-Annäherung"':}
\label{eq-disp2}
\epsilon(\vec k) &= \epsilon'_\alpha - |A| \sum_{n = 1}^6 \mathrm e^{i \vec k \vec R_n}\\
\intertext{Man erkennt aus \eqref{eq-R}, dass $R_4 = -R_1$, $R_5 = -R_2$ und $R_6 = - R_3$,
damit wird die Summe in \eqref{eq-disp2}:}
\nonumber
\sum_{n = 1}^6 \mathrm e^{i \vec k \vec R_n} &=
\sum_{n = 1}^3 \left[\mathrm e^{i \vec k \vec R_n} +  \mathrm e^{i \vec k \vec R_{n+3}} \right]
= \sum_{n = 1}^3 \left[ \mathrm e^{i \vec k \vec R_n} +  \mathrm e^{- i \vec k \vec R_{n}} \right]
= \sum_{n = 1}^3  2 \left[\cos(\vec k \vec R_n)  \right]\\
\nonumber
&= 2\left[\cos(\vec a_1 \cdot \vec k)+\cos((\vec a_1 -\vec a_2)\cdot \vec k)+\cos((\vec a_1 +\vec a_2)\cdot \vec k)\right]
\intertext{Mit $\vec k = \begin{pmatrix}k_x\\k_y\end{pmatrix}$ und \eqref{eq-a} lauten die Skalarpordukte:}
\nonumber
\sum_{n = 1}^6 \mathrm e^{i \vec k \vec R_n} &=
  2\left[\cos(ak_x)+\cos\left(\frac{a}{2}k_x -
  \frac{\sqrt{3}a}{2}k_y\right)+\cos\left(\frac{a}{2}k_x + \frac{\sqrt{3}a}{2}k_y\right)\right]
\intertext{Wegen $\cos(\alpha + \beta) + \cos(\alpha - \beta) = 2 \cos(\alpha)cos(\beta)$ ist das identisch mit}
\nonumber
\sum_{n = 1}^6 \mathrm e^{i \vec k \vec R_n} &=
  2 \left[ \cos(a k_x) + 2 \cos \left(\frac{a}{2}k_x \right)\cos \left(\frac{\sqrt{3}a}{2}k_y \right) \right]
\intertext{Eingesetzt in \eqref{eq-disp2} erhält man dann die gewünschte Dispersionsrelation:}
\nonumber
\epsilon(\vec k) &= \epsilon'_\alpha - 2 \, |A| \,
  \left[ \cos(a k_x) + 2 \cos \left(\frac{a}{2}k_x \right)\cos \left(\frac{\sqrt{3}a}{2}k_y \right) \right]
\end{align}

