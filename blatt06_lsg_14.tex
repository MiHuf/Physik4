\subsection*{Aufgabe 14}
\subsubsection*{a)}
Die reziproken Gittervektoren sind definiert durch:
\begin{align*}
\vec{b}_1 = \frac{2\pi}{V} \vec{a}_2\times\vec{a}_3\\
\vec{b}_2 = \frac{2\pi}{V} \vec{a}_3\times\vec{a}_1\\
\vec{b}_3 = \frac{2\pi}{V} \vec{a}_1\times\vec{a}_2\\
\end{align*}
Für ein fcc-Gitter mit den primitiven Basisvektoren
\begin{align*}
\vec{a}_1=\frac{a}{2}\begin{pmatrix}0\\1\\1\end{pmatrix}\\
\vec{a}_2=\frac{a}{2}\begin{pmatrix}1\\0\\1\end{pmatrix}\\
\vec{a}_3=\frac{a}{2}\begin{pmatrix}1\\1\\0\end{pmatrix}
\end{align*}
ergibt sich mit $V = \vec{a}_1 \cdot (\vec{a}_2 \times \vec{a}_3) = \frac{a^3}{4}$:
\begin{align*}
\vec{b}_1 &= \frac{2\pi}{V}\left(\frac{a}{2}\begin{pmatrix}1\\0\\1\end{pmatrix}\times\frac{a}{2}\begin{pmatrix}1\\1\\0\end{pmatrix}\right) = \frac{a^2\pi}{2V}\begin{pmatrix}-1\\1\\1\end{pmatrix} = \frac{2 \pi}{a}\begin{pmatrix}-1\\1\\1\end{pmatrix} \\
\vec{b}_2 &= \frac{2\pi}{V}\left(\frac{a}{2}\begin{pmatrix}1\\1\\0\end{pmatrix}\times\frac{a}{2}\begin{pmatrix}0\\1\\1\end{pmatrix}\right) = \frac{2\pi}{a}\begin{pmatrix}1\\-1\\1\end{pmatrix}\\
\vec{b}_3 &= \left(\frac{a}{2}\begin{pmatrix}0\\1\\1\end{pmatrix}\times\frac{a}{2}\begin{pmatrix}1\\0\\1\end{pmatrix}\right) = \frac{2\pi}{a}\begin{pmatrix}1\\1\\-1\end{pmatrix}
\end{align*}
Der Abstand $| \Gamma X |$ im reziproken Raum ist der Abstand vom Ursprung $\Gamma = (0,0,0)$ zum
Rand der 1. Brillouin-Zone $X = \frac{2 \pi}{a}(1,0,0)$, also $| \Gamma X | = \frac{2 \pi}{a}$.

\subsubsection*{b)}
Beim Vergleich mit den Energiebändern von Kupfer (s. Vorlesung) lassen sich die Hochsymmetriepunkte aus dem Verlauf der Bindungsenergie ablesen:
\begin{align*}
X &\approx 100 \unit{eV}\qquad\text{Plateau maximaler  Intensität auf der linken Seite}\\
\Gamma &\approx 176{,}5\unit{eV} \qquad\text{Tiefpunkt der Bindungsenergie}
\end{align*}
\subsubsection*{c)}
Wir berechnen die Länge des zugehörigen k-Vektors um abzuschätzen in der wievielten
Brillouin-Zone sich die gemessenen Endzustände befinden, hierfür benutzen wir die
Position des X-Punktes aus b) $E_X = 176,5 \mathrm{eV}$. Diese Länge muss dabei ein
Vielfaches des $|\Gamma X|$ Abstandes sein:


\begin{align*}
&n\cdot |\Gamma X| = |\vec{k}| = \frac{\sqrt{2mE}}{\hbar}\\
\Rightarrow & n = \frac{\sqrt{2mE}}{\hbar|\Gamma X|} = 2a\frac{\sqrt{2mE}}{\hbar\pi\sqrt{3}}\\
\Rightarrow & n =  2a\frac{\sqrt{2mE}}{\hbar\pi\sqrt{3}} = 2\cdot3{,}61\cdot10^{-10}\frac{\sqrt{2\cdot9{,}11\cdot10^{-31}\cdot176{,}5\cdot1{,}602\cdot10^{-19}}}{1{,}055\cdot10^{-34}\pi\sqrt{3}} \approx 9
\end{align*}
\subsubsection*{d)}

\subsubsection*{e)}
Bei einer mittleren freien Weglänge von $\lambda = 1\mathrm{nm}$ ist der Anteil des Signals der ersten Atomlage $I_1$ am Gesamtsignal $I_{ges}$, wobei die Ausdehnung der 1. Atomlage: $a/2 = 1{,}805\cdot10^{-10} \mathrm{m} = 0{,}1805\cdot \lambda$ sei:
\begin{align*}
\frac{I_1}{I_{ges}} = \frac{I_0\int_0^{0{,}185\lambda}\mathrm{e}^{-\frac{z}{\lambda}}\mathrm{d}z}{I_0\int_0^{\infty}\mathrm{e}^{-\frac{z}{\lambda}}\mathrm{d}z} = \frac{-\lambda(e^{-0{,}185}-1)}{\lambda} = 1-e^{-0{,}1805} \approx 0{,}165
\end{align*}
