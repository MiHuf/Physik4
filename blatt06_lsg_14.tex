\subsection*{Aufgabe 14}
\subsubsection*{a)}
Die Reziproken Gittervektoren sind definiert durch:
\begin{align*}
\vec{b}_1 = \frac{2\pi}{V} \vec{a}_2\times\vec{a}_3\\
\vec{b}_2 = \frac{2\pi}{V} \vec{a}_3\times\vec{a}_1\\
\vec{b} = \frac{2\pi}{V} \vec{a}_1\times\vec{a}_2\\
\end{align*}
Für ein fcc-Gitter mit den primitiven Basisvektoren 
\begin{align*}
\vec{a}_1=\frac{a}{2}\begin{pmatrix}0\\1\\1\end{pmatrix}\\
\vec{a}_2=\frac{a}{2}\begin{pmatrix}1\\0\\1\end{pmatrix}\\
\vec{a}_3=\frac{a}{2}\begin{pmatrix}1\\1\\0\end{pmatrix}
\end{align*}
ergibt sich damit:
\begin{align*}
\vec{b}_1 &= \frac{a}{2}\begin{pmatrix}1\\0\\1\end{pmatrix}\times\frac{a}{2}\begin{pmatrix}1\\1\\0\end{pmatrix} = \frac{2a^2\pi}{V}\begin{pmatrix}-1\\1\\1\end{pmatrix} = \frac{2\pi}{a}\begin{pmatrix}-1\\1\\1\end{pmatrix} \\
\vec{b}_2 &= \frac{a}{2}\begin{pmatrix}1\\1\\0\end{pmatrix}\times\frac{a}{2}\begin{pmatrix}0\\1\\1\end{pmatrix} = \frac{2\pi}{a}\begin{pmatrix}1\\-1\\1\end{pmatrix}\\
\vec{b}_3 &= \frac{a}{2}\begin{pmatrix}0\\1\\1\end{pmatrix}\times\frac{a}{2}\begin{pmatrix}1\\0\\1\end{pmatrix} = \frac{2\pi}{a}\begin{pmatrix}1\\1\\-1\end{pmatrix}
\end{align*}
Der Abstand $\Gamma X$ ist im reziproken Raum der Abstand von $(0,0,0)$ zu $(1,0,0)$ und somit die Länge von $\vec{b}_1$:
\begin{align*}
\Gamma x = |\vec{b}_1| = \frac{2\pi}{a}\sqrt{3}
\end{align*}

\subsubsection*{b)}

\subsubsection*{c)}

\subsubsection*{d)}

\subsubsection*{e)}
