\subsection*{Aufgabe 14}
\subsubsection*{a)}
Die Reziproken Gittervektoren sind definiert durch:
\begin{align*}
\vec{b}_1 = \frac{2\pi}{V} \vec{a}_2\times\vec{a}_3\\
\vec{b}_2 = \frac{2\pi}{V} \vec{a}_3\times\vec{a}_1\\
\vec{b} = \frac{2\pi}{V} \vec{a}_1\times\vec{a}_2\\
\end{align*}
Für ein fcc-Gitter mit den primitiven Basisvektoren 
\begin{align*}
\vec{a}_1=\frac{a}{2}\begin{pmatrix}0\\1\\1\end{pmatrix}\\
\vec{a}_2=\frac{a}{2}\begin{pmatrix}1\\0\\1\end{pmatrix}\\
\vec{a}_3=\frac{a}{2}\begin{pmatrix}1\\1\\0\end{pmatrix}
\end{align*}
ergibt sich damit:
\begin{align*}
\vec{b}_1 &= \frac{2\pi}{V}\left(\frac{a}{2}\begin{pmatrix}1\\0\\1\end{pmatrix}\times\frac{a}{2}\begin{pmatrix}1\\1\\0\end{pmatrix}\right) = \frac{a^2\pi}{2V}\begin{pmatrix}-1\\1\\1\end{pmatrix} = \frac{\pi}{2a}\begin{pmatrix}-1\\1\\1\end{pmatrix} \\
\vec{b}_2 &= \frac{2\pi}{V}\left(\frac{a}{2}\begin{pmatrix}1\\1\\0\end{pmatrix}\times\frac{a}{2}\begin{pmatrix}0\\1\\1\end{pmatrix}\right) = \frac{2\pi}{a}\begin{pmatrix}1\\-1\\1\end{pmatrix}\\
\vec{b}_3 &= \left(\frac{a}{2}\begin{pmatrix}0\\1\\1\end{pmatrix}\times\frac{a}{2}\begin{pmatrix}1\\0\\1\end{pmatrix}\right) = \frac{2\pi}{a}\begin{pmatrix}1\\1\\-1\end{pmatrix}
\end{align*}
Der Abstand $\Gamma X$ ist im reziproken Raum der Abstand von $(0,0,0)$ zu $(1,0,0)$ und somit die Länge von $\vec{b}_1$:
\begin{align*}
\Gamma x = |\vec{b}_1| = \frac{\pi}{2a}\sqrt{3}
\end{align*}

\subsubsection*{b)}
Beim Vergleich mit den Energiebändern von Kupfer (s. Vorlesung) lassen sich die Hochsymmetriepunkte aus dem Verlauf der Bindungsenergie ablesen:
\begin{align*}
\Gamma &\approx 100 \unit{eV}\qquad\text{Plateau maximaler  Intensität auf der rechten Seite}\\
x &\approx 176.5\unit{eV} \qquad\text{Tiefpunkt der Bindungsenergie}
\end{align*}
\subsubsection*{c)}
Wir berechnen die Länge des zugehörigen k-Vektors um abzuschätzen in der wievielten 
Brillouin-Zone sich die gemessenen Endzustände befinden, hierfür benutzen wir die 
Position des x-Punktes aus b) $E_x = 176.5 \mathrm{eV}$. Diese Länge muss dabei ein
vielfaches des $|\Gamma x|$ Abstandes sein:

 
\begin{align*}
&n\cdot |\Gamma x| = |\vec{k}| = \frac{\sqrt{2mE}}{\hbar}\\
\Rightarrow & n = \frac{\sqrt{2mE}}{\hbar|\Gamma x|} = 2a\frac{\sqrt{2mE}}{\hbar\pi\sqrt{3}}\\
\Rightarrow & n =  2a\frac{\sqrt{2mE}}{\hbar\pi\sqrt{3}} = 2\cdot3,61\cdot10^{-10}\frac{\sqrt{2\cdot9,11\cdot10^{-31}\cdot176.5\cdot1,602\cdot10^-19}}{1,055\cdot10^{-34}\pi\sqrt{3}} \approx 9
\end{align*}
\subsubsection*{d)}

\subsubsection*{e)}
Bei einer mittleren freien Weglänge von $\lambda = 1\mathrm{nm}$ ist der Anteil des Signals der ersten Atomlage $I_1$ am Gesamtsignal $I_ges$, wobei die Ausdehnung der 1. Atomlage: $\frac{a/2} = 1,805\cdot10^{-10} \mathrm{m} = 0,1805\cdot \lambda$ sei:
\begin{align*}
\frac{I_1}{I_ges} = \frac{I_0\int_0^{0,185\lambda}\mathrm{e}^{-\frac{z}{\lambda}}\mathrm{d}z}{I_0\int_0^{\infty}\mathrm{e}^{-\frac{z}{\lambda}}\mathrm{d}z} = \frac{-\lambda(e^{-0,185}-1)}{\lambda} = 1-e^{-0,1805} \approx 0,165
\end{align*}