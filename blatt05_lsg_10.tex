\subsection*{Aufgabe 10}
\subsubsection*{a)}
Die Phononen haben nach resonanter Anregung ebenfalls eine Frequenz von
$f_0 =$ 200 MHz, die Energie eines einzelnen Phonons ist dann $E = h f_0 =
\unit{6,626}{\cdot 10^{-34}}{\joule\second} \cdot \unit{200}{\mega\hertz} =
\unit{1,325}{\cdot 10^{-25}}{\joule}$. Wenn der Ultraschallimpuls reflexionsfrei
auf der Fläche $A = \unit{1}{\centi\meter\squared}$ eingekoppelt wird, wird die
Energie $\Delta U = I \cdot A \cdot \Delta t = \unit{1}{\milli\watt\per\centi\meter\squared}
\cdot \unit{1}{\centi\meter\squared} \cdot \unit{10}{\micro\second} =
\unit{1}{\cdot 10^{-8}}{\joule}$ auf den Kristall übertragen. Damit werden dann
$7,546 \cdot 10^{16}$ Phononen erzeugt.
\subsubsection*{b)}
Da die Temperatur $T = \unit{4,2}{\kelvin}$ klein gegen die Debye-Temperatur
$\varTheta_D = \unit{640}{\kelvin}$ ist, können wir mit akkustischen Phononen im
Debye-Modell rechnen. Für $T \ll \varTheta_D$ gilt:
\begin{align}
\label{eq-cv}
  c_V = 3 k_B N \cdot \frac{4 \pi^4}{5} \left(\frac{T}{\varTheta_D}\right)^3
\end{align}
Die würfelförmige Elementarzelle des Diamantgitters von Silizium (Volumen
$a^3 = (\unit{0,543}{\nano\meter})^3$) enthält 8 Atome (vgl. Übungsaufgabe 1.c) ),
damit ergibt sich für die Anzahl $N$ der Atome im Kristall:
\begin{align*}
  N &= 8 \cdot \frac{(\unit{1}{\centi\meter})^3} {(\unit{0,543}{\nano\meter})^3 }
  = 4,997 \cdot 10^{22}
\intertext{$N$ eingesetzt in \eqref{eq-cv} ergibt dann: }
  c_V &= \frac{12 \pi^4}{5} \cdot 4,997 \cdot 10^{22} \cdot
    \unit{1,381}{\cdot 10^{-23}}{\joule\per\kelvin}
    \cdot \left(\frac{\unit{4,2}{\kelvin}}{\unit{640}{\kelvin}} \right)^3
    = \unit{4,558}{\cdot 10^{-5}}{\joule\per\kelvin}\\
  \Delta T &= \frac{\Delta U}{c_V} =
    \frac{\unit{1}{\cdot 10^{-8}}{\joule}}{\unit{4,558}{\cdot 10^{-5}}{\joule\per\kelvin}} =
    \unit{2,194}{\cdot 10^{-4}}{\kelvin}
\end{align*}
Wegen $\Delta T \ll T$ ist es zulässig, die Temperatur-Änderung mit
konstantem $ c_V$ zu ermitteln.

\subsubsection*{c)}
In der Vorlesung (27.4.15, Folie \#10) wurde gezeigt, dass für die Zustandsdichte
$D(\omega)$ im Debye Modell gilt:
\begin{align*}
  D(\omega) &= \frac{9 \cdot N}{\omega_D^3} \cdot \omega^2
%  \qquad \text{unabhängig von $T$ solange $T \ll \varTheta_D$ gilt.}
\intertext{Mit der Debye-Frequenz $\omega_D = k_b \cdot \varTheta_D / \hbar$.
Damit wird an der Stelle $\omega_0 := 2 \pi \cdot \unit{200}{\mega\hertz}$:}
D\left(\omega_0 \right) &= \left. \frac{\mathrm d N}{\mathrm d \omega}\right|_{\omega_0} =
\frac{9 \cdot N}{\omega_D^3} \cdot \omega_0^2 \\
& = \frac{9 \cdot 4,997 \cdot 10^{22}} {\left(
\frac{\unit{1,381}{\cdot 10^{-23}}{\joule\per\kelvin} \cdot \unit{640}{\kelvin}}
{\unit{1,055}{\cdot 10^{-34}}{\joule\second}}\right)^3}
\cdot (2 \pi \cdot \unit{2}{\cdot 10^8}{\second^{-1}})^2 =
  \unit{1,207}{\second}
\intertext{oder}
\left. \frac{\mathrm d N}{\mathrm d f}\right|_{f_0} &=
  2 \pi \cdot \left. \frac{\mathrm d N}{\mathrm d  \omega}\right|_{\omega_0} =
  \unit{7,585}{\hertz^{-1}}
\end{align*}
Die Besetzungswahrscheinlichkeit $n(\omega)$ von Zuständen der Energie
$\hbar \omega_0$ ist bei Phononen (Bosonen mit Spin = 0) gegeben durch die
Bose-Einstein-Statistik:
\begin{align*}
  n(\omega_0, T) &= \frac{1}{\mathrm e^{\frac{\hbar \omega_0}{k_B T}} - 1}
  \qquad \text{bei der Frequenz $f_0$ ist }T_0 := \frac{\hbar \omega_0}{k_B} =
  \frac{h f_0}{k_B} = \unit{9,598475}{\cdot 10^{-3}}{\kelvin}
\intertext{Nach dem Temperaturanstieg um $\Delta T$ und nach Thermalisierung
ändert sich die Besetzungswahrscheinlichkeit bei der Frequenz $f_0$ um $\Delta n$:}
  \Delta n &= n(\omega_0, T + \Delta T) -  n(\omega_0, T) =
  \frac{1}{\mathrm e^{\frac{T_0}{T + \Delta T}} - 1} -
  \frac{1}{\mathrm e^{\frac{T_0}{T}} - 1}
\intertext {mit den Zahlenwerten $T = \unit{4,2}{\kelvin}$ und
$\Delta T = \unit{2,194}{\cdot 10^{-4}}{\kelvin}$ ergibt sich:}
  \Delta n &= 0,0229 = 2,29 \%
\end{align*}
Damit ändert sich die Besetzungszahl $\Delta Z$ bei $f_0$ gemäß
\begin{align*}
  \Delta Z = D(f_0) \cdot \Delta n =
  \unit{7,585}{\hertz^{-1}} \cdot 0,0229 = \unit{0,18}{\hertz^{-1}}
\end{align*}





