\subsection*{Aufgabe 10}
\subsubsection*{a)}
Die Phononen haben nach resonanter Anregung ebenfalls eine Frequenz von
$f =$ 200 MHz, die Energie eines einzelnen Phonons ist dann $E = h f =
\unit{6,626}{\cdot 10^{-34}}{\joule\second} \cdot \unit{200}{\mega\hertz} =
\unit{1,325}{\cdot 10^{-25}}{\joule}$. Wenn der Ultraschallimpuls reflexionsfrei
auf der Fläche $A = \unit{1}{\centi\meter\squared}$ eingekoppelt wird, wird die
Energie $E = I \cdot A \cdot t = \unit{1}{\milli\watt\per\centi\meter\squared}
\cdot \unit{1}{\centi\meter\squared} \cdot \unit{10}{\micro\second} =
\unit{1}{\cdot 10^{-8}}{\joule}$ auf den Kristall übertragen. Damit werden dann
$7,546 \cdot 10^{16}$ Phononen erzeugt.
\subsubsection*{b)}
Da die Temperatur $T = \unit{4,2}{\kelvin}$ klein gegen die Debye-Temperatur
$\varTheta_D = \unit{640}{\kelvin}$ ist, können wir mit akkustischen Phononen im
Debye-Modell rechnen. Für $T \ll \varTheta_D$ gilt:
\begin{align}
\label{eq-cv}
  c_V = 3 k_B N \cdot \frac{4 \pi^4}{5} \left(\frac{T}{\varTheta_D}\right)^3
\end{align}
Die würfelförmige Elementarzelle des Diamantgitters von Silizium (Volumen
$a^3 = (\unit{0,543}{\nano\meter})^3$) enthält 8 Atome (vgl. Übungsaufgabe 1.c) ),
damit ergibt sich für die Anzahl $N$ der Atome im Kristall:
\begin{align*}
  N &= 8 \cdot \frac{(\unit{1}{\centi\meter})^3} {(\unit{0,543}{\nano\meter})^3 }
  = 4,997 \cdot 10^{22}
\intertext{$N$ eingesetzt in \eqref{eq-cv} ergibt dann: }
  c_V &= \frac{12 \pi^4}{5} \cdot 4,997 \cdot 10^{22} \cdot
    \unit {1,381}{\cdot 10^{-23}}{\joule\per\kelvin}
    \cdot \left(\frac{\unit{4,2}{\kelvin}}{\unit{640}{\kelvin}} \right)^3
    = \unit{4,56}{\cdot 10^{-5}}{\joule\per\kelvin}\\
  \Delta T &= \frac{\Delta U}{c_V} =
    \frac{\unit{1}{\cdot 10^{-8}}{\joule}}{\unit{4,56}{\cdot 10^{-5}}{\joule\per\kelvin}} =
    \unit{2,19}{\cdot 10^{-4}}{\kelvin}
\end{align*}
Wegen $\Delta T \ll T$ ist es zulässig, die Temperatur-Änderung mit
konstantem $ c_V$ zu ermitteln.

\subsubsection*{c)}
