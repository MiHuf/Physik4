% DOCUMENT CLASS
\documentclass[11pt]{article}
%PACKAGES
\usepackage[utf8]{inputenc}
\usepackage[ngerman]{babel}
\usepackage[reqno,fleqn]{amsmath}
\setlength\mathindent{10mm}
\usepackage{amstext}
\usepackage{amssymb}
\usepackage{fancyhdr}
% Zahlenwerte mit Einheiten mittels \unit{Zahlenwert}{Einheit}
\usepackage[thinspace,thinqspace,squaren,textstyle]{SIunits}
% FORMATIERUNG
\usepackage[paper=a4paper,left=25mm,right=25mm,top=25mm,bottom=25mm]{geometry}
\setlength{\parindent}{0cm}
\setlength{\parskip}{1.5mm plus1mm minus1.5mm}
% PAGESTYLE
\pagestyle{fancy}
\setlength\headheight{30pt}
\lhead{Michael Hufschmidt, Mat. Nr. 6436122\\Florian Jochheim, Mat. Nr. 6508131}
\rhead{Übungen zur Physik IV, SoSe 2015\\Blatt 03 zum 27.04.2015}
%MATH SHORTCUTS
\newcommand*{\NN}{\mathbb N}


\begin{document}
\subsection*{Aufgabe 6}
\subsubsection*{a)}
Bei der Steuung an Kristallen ist die Streuamplitude $A$ gegeben durch
$A \propto N_E \cdot S_{\vec G}$ mit $N_E$ als Zahl der Elementarzellen und
dem Strukturfaktor $S_{\vec G}$. Dieser ergibt sich durch Integration über die
Elektronendichte $\rho_A^j$ des $j$-ten Atoms einer Elementarzelle und durch
Summation über die Atome einer Elementarzelle gemäß
\begin{align*}
  S_{\vec G} = \sum_{j} e^{i \vec G \cdot \vec r_j} \cdot \underbrace{
  \left[ \int_{\text{Atom}}\rho_A^j (\vec r') d^3 r'  \right] }_{= f_j} =
  \sum_{j} f_j \cdot e^{i \vec G \cdot \vec r_j}
\end{align*}
Da bei einem Elementkristall alle Atome identisch sind, sind auch alle $f_j$
identisch $= f$. Der Vektor $\vec r_j$ zeigt stets auf einen Punkt
im Bravais-Gitter: $\vec r = u_j \vec a_1 + v_j \vec a_2 + w_j \vec a_3$ mit
ganzen Zahlen $u_j, v_j, w_j$. Die Laue-Bedingung bedeutet, dass
$\Delta \vec k = \vec G = h \vec b_1 + k \vec b_2 + l \vec b_3$ ein Gitterpunkt
im reziproken Gitter ist. Mit der Definition der Basisvektoren des reziproken
Gitters bedeutet das $\vec G \cdot \vec r_j =  (u_j \vec a_1 + v_j \vec a_2 + w_j \vec a_3)
\cdot (h \vec b_1 + k \vec b_2 + l \vec b_3) = 2 \pi (h u_j + k v_j + l w_j)$
Die Elementarzelle im kubisch flächenzetrierten Gitter hat Gitterpunkte bei
$(u_j,v_j,w_j) = (0,0,0); (0,\frac{1}{2},\frac{1}{2}); (\frac{1}{2}, 0, \frac{1}{2});
(\frac{1}{2},\frac{1}{2}, 0)$. Damit wird
\begin{align*}
  S_{\vec G} = f \cdot \sum_{j}  e^{i \vec G \cdot \vec r_j} =
  f \cdot \left(1 + e^{i \pi (h + k)} + e^{i \pi (h + l)} +e^{i \pi (k + l)} \right)
  \text{ mit } h, k, l \in \NN
\end{align*}
Die Exponentialfunktionen haben entweder den Wert $+1$ oder $-1$, insgesamt wird dann
wenn alle $h, k, l$ gerade oder alle ungerade sind $S_{\vec G} = 4 f$, ansonsten gilt
$S_{\vec G} = 0$.

Da im Streuspektrum Linien gerader und ungerader Ebenen gemischt vorkommen
(1,1,0) und (2,1,1), kann es sich nicht um einen fcc-Kristall handeln,
gemäß Aufgabenstellung bleibt dann nur noch eine gcc-Struktur für das Gitter übrig.





\subsubsection*{b)}

\subsubsection*{c)}

\subsection*{Aufgabe 7}
\subsubsection*{a)}

\subsubsection*{b)}

\subsubsection*{c)}

\end{document}

