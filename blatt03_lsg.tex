% DOCUMENT CLASS
\documentclass[11pt]{article}
%PACKAGES
\usepackage[utf8]{inputenc}
\usepackage[ngerman]{babel}
\usepackage[reqno,fleqn]{amsmath}
\setlength\mathindent{10mm}
\usepackage{amstext}
\usepackage{amssymb}
\usepackage{fancyhdr}
% Zahlenwerte mit Einheiten mittels \unit{Zahlenwert}{Einheit}
\usepackage[thinspace,thinqspace,squaren,textstyle]{SIunits}
% FORMATIERUNG
\usepackage[paper=a4paper,left=25mm,right=25mm,top=25mm,bottom=25mm]{geometry}
\setlength{\parindent}{0cm}
\setlength{\parskip}{1.5mm plus1mm minus1.5mm}
% PAGESTYLE
\pagestyle{fancy}
\setlength\headheight{30pt}
\lhead{Michael Hufschmidt, Mat. Nr. 6436122\\Florian Jochheim, Mat. Nr. 6508131}
\rhead{Übungen zur Physik IV, SoSe 2015\\Blatt 03 zum 27.04.2015}
%MATH SHORTCUTS
\newcommand*{\NN}{\mathbb N}


\begin{document}
\subsection*{Aufgabe 6}
\subsubsection*{a)}
Bei der Steuung an Kristallen ist die Streuamplitude $A$ gegeben durch
$A \propto N_E \cdot S_{\vec G}$ mit $N_E$ als Zahl der Elementarzellen und
dem Strukturfaktor $S_{\vec G}$. Dieser ergibt sich durch Integration über die
Elektronendichte $\rho_A^j$ des $j$-ten Atoms einer Wigner-Seitz-Zelle (WS)
und durch Summation über die Atome der Wigner-Seitz-Zelle gemäß
\begin{align*}
  S_{\vec G} = \sum_{j \in \text{WS}} e^{i \vec G \cdot \vec r_j} \cdot \underbrace{
   \int \limits_{\text{Atom}}\rho_A^j (\vec r') d^3 r'}_{= f_j} =
   \sum_{j \in \text{WS}} f_j \cdot e^{i \vec G \cdot \vec r_j}
\end{align*}
Da bei einem Elementkristall alle Atome identisch sind, sind auch alle $f_j$
identisch $= f$. Der Vektor $\vec r_j$ zeigt stets auf einen Punkt
im Bravais-Gitter: $\vec r = u_j \vec a_1 + v_j \vec a_2 + w_j \vec a_3$ mit
ganzen Zahlen $u_j, v_j, w_j$. Die Laue-Bedingung bedeutet, dass
$\Delta \vec k = \vec G = h \vec b_1 + k \vec b_2 + l \vec b_3$ ein Gitterpunkt
im reziproken Gitter ist. Mit der Definition der Basisvektoren des reziproken
Gitters bedeutet das $\vec G \cdot \vec r_j =  (u_j \vec a_1 + v_j \vec a_2 + w_j \vec a_3)
\cdot (h \vec b_1 + k \vec b_2 + l \vec b_3) = 2 \pi (h u_j + k v_j + l w_j)$
Die Elementarzelle im kubisch flächenzetrierten Gitter hat Gitterpunkte bei
$(u_j,v_j,w_j) = (0,0,0); (0,\frac{1}{2},\frac{1}{2}); (\frac{1}{2}, 0, \frac{1}{2});
(\frac{1}{2},\frac{1}{2}, 0)$. Damit wird
\begin{align*}
  S_{\vec G} = f \cdot \sum_{j}  e^{i \vec G \cdot \vec r_j} =
  f \cdot \left(1 + e^{i \pi (h + k)} + e^{i \pi (h + l)} +e^{i \pi (k + l)} \right)
  \text{ mit } h, k, l \in \NN
\end{align*}
Die Exponentialfunktionen haben entweder den Wert $+1$ oder $-1$, insgesamt wird dann
wenn alle $h, k, l$ gerade oder alle ungerade sind $S_{\vec G} = 4 f$, ansonsten gilt
$S_{\vec G} = 0$.

Da im Streuspektrum Linien gerader und ungerader Ebenen gemischt vorkommen
(1,1,0) und (2,1,1), kann es sich nicht um einen fcc-Kristall handeln,
gemäß Aufgabenstellung bleibt dann nur noch eine bcc-Struktur für das Gitter übrig.

\subsubsection*{b)}
Die Netzebenenabstände lassen sich aus der Bragg-Bedingung
$2 \cdot d_{hkl} \cdot \sin(\varTheta) = n \cdot \lambda$ bzw. für die erste Ordnung
$d_{hkl} = \frac{\lambda}{2 \sin(\varTheta)}$ bestimmen. Mit den Zahlenwerten aus dem
Spektrum und $\lambda =\unit{154}{\pico \meter}$\footnote{Die Einheit $\mathrm{\AA}$
ist in DIN 1301-3 ausdrücklich als nicht mehr zugelassene Einheit aufgeführt.
Ihre Verwendung wurde im "`Gesetz über die Einheiten im Messwesen und die
Zeitbestimmung (Einheiten- und Zeitgesetz - EinhZeitG)"' von 1969 im
geschäflichen Verkehr verboten und kann als Ordnungswidrigkeit mit
Bußgeldern geahndet werden. Sie sollte daher auch im Lehrbetrieb nicht verwendet werden.}
wird $d_{hkl} = \frac{\unit{77}{\pico \meter}}{\sin(\varTheta)}$.
Damit ergeben sich die Netzebenenabstände für die 4 Peaks gemäß folgender Tabelle:
\begin{table}[htbp]
% \caption{}
\begin{center}
\begin{tabular}{|c|c|r|r|r|}
\hline
Peak & $h,k,l$ & $ 2 \cdot \varTheta / { }^\circ $ & $ \varTheta / { }^\circ $
  & $d_{hkl}$ / pm \\ \hline \hline
\#1 & 1,1,0 & 44,67 & 22,335 & 202,6 \\ \hline
\#2 & 2,0,0 & 65,02 & 32,510 & 143,3 \\ \hline
\#3 & 2,1,1 & 82,33 & 41,165 & 117,0 \\ \hline
\#4 & 2,2,0 & 98,94 & 49,470 & 101,3 \\ \hline
\end{tabular}
\end{center}
\label{}
\end{table}

\subsubsection*{c)}
Die Netzebene (2,0,0) für Peak \#2 schneidet die x-Achse im Punkt
$\frac{1}{2} \cdot a$, daraus ergibt sich eine Gitterkonstante
$a = \unit{286,6}{\pico \meter}$, es handelt sich also vermutlich um
\textbf{Eisen}: bcc-Struktur mit $a = \unit{287}{\pico \meter}$.

\subsection*{Aufgabe 7}
\subsubsection*{a)}

\subsubsection*{b)}

\subsubsection*{c)}

\end{document}

